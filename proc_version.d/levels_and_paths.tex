\section{DAG-SyDSs: Some Phase Space Properties}
\label{sec:bounded-levels}

In this section, we present some properties of the
phase spaces of DAG-SyDSs.
These properties are independent of the local functions
at the nodes of the DAG-SyDS.
Our first result points out that the phase spaces
of DAG-SyDSs may contain exponentially large cycles.

\smallskip
\begin{proposition}\label{pro:long_phase_space_cycle}
For any $n  > 1$, there is an $n$ node DAG-SyDS 
whose phase space graph is a cycle of length $2^n$.
\end{proposition}

\noindent
\textbf{Proof:}~ 
For a given $n > 1$ we construct the DAG-SyDS $S_n$ to be a counter, as follows.
The underlying graph contains $n$ levels, one node per level. 
For each node, there is an incoming edge from the nodes on each of the lower levels.
The transition function for each node is the function 
that retains the current value of the node if any of the lower order bits is 0,
and changes the value of the node if all of the lower order bits are 1.

Suppose a given configuration of $S_n$ is interpreted as encoding
an integer $k$, $0 \leq k < 2^n$.  Then the successor configuration
encodes the integer $k + 1 \mod 2^n$.  Thus, the phase space of
$S_n$ is a cycle of length $2^n$.  \QED

\smallskip

For any SyDS, any infinitely long phase space path consists of a
transient (possibly of length 0), followed by an infinite number
of repetitions of a basic cycle.  We now show that for any 
DAG-SyDS, the length of every phase space cycle is a power of 2.
Moreover, the lengths of the longest transient and the longest phase
space cycle are each bounded by an exponential function of the
number of levels in the underlying graph of the SyDS.
We begin with a lemma whose proof appears in \cite{Rosenkrantz-etal-2020}.

\smallskip

\begin{lemma}\label{lem:all_inputs_stable}
For a given DAG-SyDS, and a given initial configuration, suppose
that the node values of all incoming edges to a given node $v$ are
stable at time $t$.  Then node $v$ is either alternating at time
$t$, or stable at time $t+1$.
\end{lemma}

\begin{lemma}\label{lem:level_zero_nodes}
For a DAG-SyDS, every level 0 node is either
alternating at time 0 or stable at time 1.
\end{lemma}
\noindent
\textbf{Proof:}~
 A level 0 node has no incoming edges,
so the result follows from Lemma \ref{lem:all_inputs_stable}.  \QED

We will also use the following result which is a slight restatement
of Proposition~1 in \cite{Chistikov-etal-2020}.


\begin{proposition}\label{pro:transient_fixed_point}
For a DAG-SyDS,
the length of a transient leading to a fixed point does not
exceed the number of levels of the SyDS.
\end{proposition}

We can now state our result on the lengths of cycles and
transients in DAG-SyDSs.

\begin{theorem}\label{thm:levels_phase_space}
For a DAG-SyDS,
the length of every phase space cycle is a power of 2.
Moreover, if the number of levels of a given acyclic SyDS is $L$,
then no phase space cycle is longer than $2^L$,
and no transient is longer than $2^L-1$.
\end{theorem}

\noindent
\textbf{Proof (idea):}~ We use induction on the number of levels.
The details appear in \cite{Rosenkrantz-etal-2020}.

Theorem~\ref{thm:levels_phase_space} provides upper bounds
on the lengths of transients and cycles in DAG-SyDSs.
We now present a result that provides matching
lower bounds on cycle and transient lengths.

\begin{theorem}\label{thm:path_length_lower_bounds}
For every $L  \geq 1$, there is a DAG-SyDS with $L$ levels
whose phase space contains a transient of length $2^L-1$,
leading to a cycle of length $2^L$.
\end{theorem}

\noindent
\textbf{Proof:}~ See \cite{Rosenkrantz-etal-2020}.
