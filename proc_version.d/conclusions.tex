\section{Directions for  Future Work}
\label{sec:concl}
 
%We considered DAG-SyDSs 
%and established some properties of the phase spaces of
%such dynamical systems. We used those properties in proving
%complexity results and developing algorithms 
%for problems such as reachability and convergence guarantee.
We conclude by mentioning two directions for future research.
For example, it will be of interest to consider 
DAG-SyDSs where local functions are from other classes
(e.g., weighted threshold functions \cite{Crama-Hammer-2011}).
%and establish properties of phase spaces.
Similarly, one may also consider restrictions
on the graph structure (e.g., dags where nodes have
bounded in-degrees) and investigate 
the complexity of reachability and other problems.
%As another extension, it is also of interest to investigate
%synchronous dynamical systems over directed graphs which are
%near-acylic in some fashion (e.g., removing a small
%number of directed edges leads to a dag).


\iffalse
%%%%%%%%%%%%%%%%%%%%%%%%%%
\noindent
\paragraph{Conclusions.}


\paragraph{Open Questions.}

\begin{itemize}
\item For DAG linear SyDSs, how long can a phase space cycle and transient be,
as a function of depth.

\item For bounded degree DAG  SyDSs, how long can a phase space cycle and transient be,
as a function of depth.

\item For various classes of Boolean functions of interest, 
how long can a phase space cycle and transient be?
(If they are polynomially bounded as a function of the number of nodes,
then reachability is polynomially solvable.)
\end{itemize}
%%%%%%%%%%%%%%%%%%%%%%%%%%
\fi
