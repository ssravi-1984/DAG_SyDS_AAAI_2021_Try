\subsection{Reducing the Number of Symmetry Classes}
\label{sse:embedding}

In this section, we show how the reachability problem for
$r$-symmetric SyDSs (for any fixed $r$)
can be reduced to the same problem for symmetric SyDSs.
In particular, this reduction ensures that when we start
with a DAG-SyDS, the reduction also produces a DAG-SyDS.
Our approach uses the idea of embedding, which is defined below.

\begin{definition}\label{def:SyDS_homomorphisml}
Given a pair of SyDSs $\cals = (G(V,E), \calf)$ and $\cals' = (G'(V',E'), \calf')$,
let $h$ be an onto function from $V'$ to $V$.
Let $\hat{h}$ be the function that maps each configuration $\calc$ of $\cals$
into the configuration $\calc'$ of $\cals'$ 
such that for each node $v' \in V'$, 
$\calc'(v') = \calc(h(v'))$.
We say that $h$ is an \textbf{embedding} of $\cals$ into $\cals'$
if for every configuration $\calc$ of $\cals$,
letting $\cald$  be the configuration such that $\calc \longrightarrow \cald$ in $\cals$,
$\hat{h}(\calc) \longrightarrow \hat{h}(\cald)$ in $\cals'$.
\end{definition}

\noindent
The next lemma shows a property of an embedding.

\begin{lemma}\label{lem:embedding_reachability}
 Given SyDSs $\cals$ and $\cals'$, and an embedding $h$ from $\cals$ into  $\cals'$,
the reachability problem for $\cals$ is polynomially reducible
to the reachability problem for $\cals'$.
\end{lemma}

\noindent
\textbf{Proof:}~
In Definition~\ref{def:SyDS_homomorphisml},
since $h$ is an onto function, $\hat{h}$  is injective.
For every configuration $\calc$ of $\cals$ and every $t \geq 0$,
by induction on $t$,
$\hat{h}(\calc_t) = (\hat{h}(\calc))_t $.
Thus, $\calc$ reaches a given configuration $\cald$
iff $\hat{h}(\calc)$ reaches the configuration $\hat{h}(\cald)$.
\QED

\begin{definition}\label{tdef:generalized_neighbor}
A {\bf generalized neighbor} of a given node $v$ in a directed graph is a node that
is either the source node of an incoming edge to $v$, or the node $v$  itself.
\end{definition}

We can now state a result which shows that a suitable embedding
can be constructed efficiently.
%For space reasons, its proof appears in the supplement.

\begin{theorem}\label{thm:r_symmetric_directed}
For a fixed value of $r$, there is a polynomial time algorithm that
given a directed $r$-symmetric SyDSs $\cals = (G(V,E), \calf)$,
constructs a directed symmetric SyDSs $\cals' = (G'(V',E'), \calf')$
and an embedding $h$ of $\cals$ into $\cals'$.  Moreover, if  $G$
is acyclic, then so is $G'$.  
\end{theorem}

\noindent
\textbf{Proof:}~ See supplement.

\iffalse
%%%%%%%%%%%%%%%%%%%%%%%%%%%%%%
\noindent
\textbf{Proof:}~ 
For a given node $v \in V$, 
suppose that local transition function $f_v$ contains $m^v$ classes.
Note that $1 \leq m^v \leq r$.
Let $\nu_0^v, \nu_1^v, \ldots ,\nu_{m^v-1}^v$ be these classes,
where $\nu_0^v$ is the class that contains self variable $v$.

We assume that function $f_v$  is represented 
as a $m^v$-dimensional table,
accompanied by a list of which nodes are in each class.
This table gives the value of $f_v$ for each tuple
of the form $j_0,  j_1, \ldots , j_{m^v -1}$,
where for each $i$, $0 \leq j_i \leq |\nu_i^v|$.

We construct $\cals'$ as follows.
Let $\Delta$ be the cardinality of the largest class in $\calf$.
Let $q =  \Delta + 1$.

Node set $V'$ and function $h$ are constructed as follows.
For each node $v \in V$,
node set $V'$ contains $q^{r-1}$ nodes, each of which is mapped into $v$ by $h$.

 Edge set $E'$ is constructed as follows.
For a given node $v' \in V'$,
suppose that $h(v') = v$.
Consider an incoming edge $(u,v) \in E$.
Suppose that $u \in \nu_i^v$.
Then $q^i$ nodes in $h^{-1}(u)$ are selected,
and an incoming edge to $v'$ from each of these nodes is added to $E'$.
Note that $v$  lies in class  $\nu_0^v$,
so no incoming edges are added from nodes in $h^{-1}(v) - \{v'\}$.
Thus, if $G$ is a DAG, then so is $G'$.

The set of symmetric local transition functions $\calf'$ is constructed as follows.
Consider a given node $v' \in V'$. Let $v = h(v')$.
Let $c_v = \sum_{i=0}^{m^v-1} q^i \, |\nu_i^v|$.
Note that the number of incoming edges into $v'$ is 
$c_v -1$.
Thus, the table specifying symmetric function $f_{v'}$
has an entry for each value $j$ such that $0 \leq j \leq c_v$.
Consider the entry for a given $j$.
For each $i$ such that $0 \leq i < m_v$,
we compute values $k_i$ and $j_i$ as follows.
We set $k_0 =j$, and then for $1 \leq i < m_v$,
set $k_i = \lfloor k_{i-1}/q \rfloor$.
Then, for $0 \leq i < m_v$, set $j_i = k_i \mod q$.
If each $j_i$ satisfies the requirement that $j_i \leq | \nu_i^v |$,
we let $\hat{j}$ be the $m^v$-tuple $j_0,  j_1, \ldots , j_{m^v -1}$,
and set the table entry for $j$ in $f_{v'}$ equal to the entry for $\hat{j}$ 
 in the table for $f_v$.
Otherwise we set the table entry for $i$ in $f_{v'}$ equal to an arbitrary value.
Note that all nodes $v'$ such that $h(v')=v$ have the same table.

Since $r$ is fixed, the above construction can be done in polynomial time.

We now argue that $h$ is indeed an embedding of $\cals$ into $\cals'$.
Consider a given configuration $\calc$ of $\cals$.
Let $\cald$  be the configuration such that $\calc \longrightarrow \cald$.
Let $\calc' =\hat{h}(\calc)$, 
and let $\cald$  be the configuration such that $\calc' \longrightarrow \cald'$.

Consider a given node $v' \in V'$.
Suppose that $h(v') = v$.
Suppose that for $0 \leq i < m^v$,
class $\nu_i^v$ contains $j_i$ generalized neighbors of $v$ that are 1 in $\calc$.
Then $\cald(v) = f_v(j_0,  j_1, \ldots , j_{m^v -1})$.
Consider a given class $\nu_i^v$ of $f_v$.
Let $u$ be a generalized neighbor of $v$ such that $ u \in \nu_i^v$.
Then node $v'$ has $q^i$ generalized neighbors $u'$ such that $h(u') = u$.
Since $\calc' =\hat{h}(\calc)$, for each of these $q^i$ nodes $u'$,
$\calc'(u') = \calc(u)$.
Since the number of generalized neighbors of $v$ from class $\nu_i^v$ 
with value 1 in $\calc$ is $j_i$,
the number of generalized neighbors of $v'$ 
that map into members of $\nu_i^v$ under $h$ and have value 1 in $\calc'$ is $q^i j_i$.
Considering all the classes of $f_v$,
the number of generalized neighbors of $v'$ that have value 1 in $\calc'$ is 
$j = \sum_{i=0}^{m^v-1} q^i \, j_i$.
But, by the construction of $\cals'$,
$f_{v'}(j) = f_v(j_0,  j_1, \ldots , j_{m^v -1})$.
Thus, $\cald'(v') = \cald(v)$.

Since the above applies to every node $v' \in V'$,  $\cald' =\hat{h}(\cald)$.
Thus, $h$ is an embedding of $\cals$ into $\cals'$.
\QED
%%%%%%%%%%%%%%%%%%%%%%%%%%%%%%
\fi

\smallskip

The following is the main result of Section~\ref{sse:embedding}.

\begin{theorem}\label{thm:reachability_symmetric}
For a fixed value of $r$, the reachability problem for directed
$r$-symmetric  SyDSs is polynomial-time reducible to the reachability
problem for directed symmetric SyDSs, and  the reachability problem
for DAG $r$-symmetric  SyDSs is polynomial-time reducible to the
reachability problem for DAG  symmetric  SyDSs.  
\end{theorem}

\noindent 
\textbf{Proof:}~ From Lemma~\ref{lem:embedding_reachability}
and Theorem~\ref{thm:r_symmetric_directed}.  \QED

\smallskip

Theorems~\ref{thm:reach-six-sym} and \ref{thm:reach-six-sym}
together imply the following main result
of Section~\ref{sec:reachability}.

\begin{theorem}\label{thm:dag_syds_sym_hard}
The Reachability problem for DAG-SyDS with symmetric local functions is
\cpsp-complete. \QED
\end{theorem}

