\section{Introduction}
\label{sec:intro}

\subsection{Motivation}
\label{sse:motivation}

\textcolor{red}{(To be revised)}

We consider directed SyDSs whose underlying graph is a directed
acyclic graph (dag).  We call such a SyDS a {\bf DAG SyDS}.  For
any class of local transition functions $\Gamma$, we call a dag
SyDS whose local transition functions are all in  $\Gamma$, a {\bf
DAG $\Gamma$ SyDS}.  We also call a SyDS whose underlying graph is
directed (but not necessarily acyclic) a {\bf directed SyDS}.


We consider phase space properties of DAG  SyDSs, as well as the
complexity of several computational problems.  We also study these
issues for various classes of DAG  SyDSs defined by restrictions
on the underlying dag of the SyDS and/or on the local transition
functions.

For a given SyDS, the length of  phase space paths is related to
the reachability problem.  If a given class of SyDSs has a polynomial
bound on the length of the longest phase space cycle and longest
transient, then this bound applies to the number of transitions
required to solve the reachability problem for this class of SyDSs.

%\bigskip

\noindent
\textbf{Summary of Results and Their Significance:}

\smallskip

\noindent
\underline{(1) Results on the structure of the phase space:}
For any DAG-SyDS, we show that
the length of every phase space cycle is a power of 2.
If the number of levels in the underlying DAG of a given SyDS is $L$,
then no phase space cycle is longer than $2^L$,
and no transient is longer than $2^L-1$
(Theorem \ref{thm:levels_phase_space}).
Moreover, for each $L$, we observe that
there is an $L$-level DAG SyDS whose phase 
space graph is a cycle of length $2^L$
(Proposition \ref{pro:long_phase_space_cycle}).
We use these properties to establish some of
the results in subsequent sections.
As discussed in Item~(3) below, we establish more refined versions
of these properties for DAG-SyDSs whose local functions are from
special classes of Boolean functions.

\smallskip

\noindent
\underline{(2) Complexity of the Reachability Problem for DAG-}\newline
\underline{SyDSs:} 
It was shown in
\cite{Chistikov-etal-2020} that the Convergence problem
(i.e., given a SyDS \cals{} and an initial configuration,
does \cals{} reach a fixed point?) is \cpsp-complete
for SyDSs on directed networks.
They also showed that the convergence problem can be solved efficiently
for DAG-SyDSs.
We show that the reachability problem
(i.e., given a SyDS \cals{} and two configurations $\cali_1$ and $\cali_2$,
does \cals{} starting from $\cali_1$ reach $\cali_2$?),
which is closely related
to the Convergence problem, is \cpsp-complete for DAG-SyDSs even
when each local function is symmetric.
(Please see Section~\ref{sec:defs} for definitions of various classes
of Boolean functions.)
It was shown in \cite{OU-2017} that the Reachability
problem is \cpsp-complete for SyDSs on directed graphs 
which may contain cycles.
To our knowledge, no complexity result is currently known for
the Reachability problem for DAG-SyDSs.
Our proof of this result involves two major steps.
The first step uses a reduction from the \textbf{Quantified Boolean Formulas}
(QBF) problem \cite{GJ-1979} to show that the Reachability problem for DAG-SyDSs is
\cpsp-complete even when each local function is $r$-symmetric for 
some constant $r$ (Theorem~\ref{thm:reachability-PSPACE}).
For the second step,
we define the concept of an \textbf{embedding} of one SyDS into another,
and use this concept to show that
for any fixed value of $r$, 
the reachability problem for DAG-SyDSs with $r$-symmetric local functions 
is polynomial-time reducible to the Reachability problem for DAG-SyDSs with
symmetric  local functions (Corollary \ref{cor:reachability_symmetric}).

\smallskip

\noindent
\underline{(3) Efficient Solvability of Reachability Problem for Special}\newline
\underline{Classes of DAG-SyDSs:} 
We show that the Reachability problem is efficiently solvable for
DAG-SyDSs whose local functions are from two specific
classes of Boolean functions, namely monotone functions and
nested canalyzing functions (NCFs).
For any DAG-SyDS whose local functions are monotone,
we show that every cycle is a fixed point, 
and the length of any transient does not exceed the number of levels
(Theorem \ref{thm:montone_stable}).
For a DAG-SyDS \cals{} whose local functions are NCFs, we show
that the length of every phase space cycle is at most 2.
Moreover, if the number of levels of  \cals{} is $L$,
no transient is longer than $2L-1$
(Theorem \ref{thm:NCF_phase_space}).
These properties
directly imply the efficient solvability of the reachability
problem for such DAG-SyDSs.
It is known that the Reachability problem for SyDSs whose local
functions are NCFs and whose directed graphs which may contain cycles  
is \cpsp-complete \cite{Rosenkrantz-etal-2018}.
To our knowledge, the only previous result on the efficient
solvability of the  Reachability problem for DAG-SyDSs is
for the case when each local function is a bithreshold
function \cite{KKM+2013}.

\smallskip

\noindent
\underline{(4) Results for the Convergence Guarantee Problem:}
We show that
the Convergence Guarantee problem (i.e., given a SyDS \cals{} on
a directed graph, does \cals{} reach a fixed point from every
initial configuration?)  for DAG SyDSs is \cconp-complete,
even when restricted to dags with  three levels and 3-symmetric functions
(Theorem \ref{thm:convergence_guaranee}).
It was shown in \cite{Chistikov-etal-2020} that
the Convergence Guarantee problem is \cpsp-complete
for SyDSs on general directed graphs.
Our result points out that a different computational intractability
result holds for the problem even for DAG-SyDSs.

\iffalse
%%%%%%%%%%%%%%%%%%%%%%%%%%%
\item \textbf{Results for Garden of Eden, Fixed Point and Predecessor Existence
Problems:}

\begin{description}
\item{(a)} We define the concept of a local transition function being {\bf balanced},
and relate this concept to the Garden of Eden problem for DAG SyDSs.

\item{(b)} For fixed $r$, the GE existence problem for DAG SyDSs
whose local transition functions are specified as $r$-symmetric tables
can be solved in polynomial time
(Theorem \ref{thm:GardenEden-r_symmetric}).

\item{(c)} The GE existence problem for DAG SyDSs
whose local transition functions are specified as Boolean formulas
is NP-complete,
even when restricted to DAGs with two levels
(Theorem \ref{thm:GardenEden-NP}).

\item{(d)} The problem of determining whether a given DAG SyDS 
has a fixed point is NP-complete,
even when restricted to dags with two levels and maximum node degree 3
(Theorem \ref{thm:fixed_point}).

\item{(e)} The predecessor existence problem for DAG SyDSs  is NP-complete,
even when restricted to dags with two levels and maximum node degree 3
(Theorem \ref{thm:predecessor_existence}).
\end{description}
%%%%%%%%%%%%%%%%%%%%%%%%%%%%
\fi

\subsection{Related Work}
\label{sse:related}

In our earlier paper on Bi-Threshold SyDSs \cite{KKM+2013}, 
we showed that for DAG bi-threshold SyDSs,
every phase space cycle is of length at most two, 
and the length of a transient is at most $2L-1$,
with the consequence that reachability can be solved in polynomial time.
We also showed that the predecessor existence problem is NP-complete, 
even when all threshold values are 1, and there are only three levels in the dag.
