\section{Introduction}
\label{sec:intro}

\subsection{Motivation}
\label{sse:motivation}


We consider directed SyDSs whose underlying graph is a directed
acyclic graph (dag).  We call such a SyDS a {\bf DAG SyDS}.  For
any class of local transition functions $\Gamma$, we call a dag
SyDS whose local transition functions are all in  $\Gamma$, a {\bf
DAG $\Gamma$ SyDS}.  We also call a SyDS whose underlying graph is
directed (but not necessarily acyclic) a {\bf directed SyDS}.


We consider phase space properties of DAG  SyDSs, as well as the
complexity of several computational problems.  We also study these
issues for various classes of DAG  SyDSs defined by restrictions
on the underlying dag of the SyDS and/or on the local transition
functions.

For a given SyDS, the length of  phase space paths is related to
the reachability problem.  If a given class of SyDSs has a polynomial
bound on the length of the longest phase space cycle and longest
transient, then this bound applies to the number of transitions
required to solve the reachability problem for this class of SyDSs.

%\bigskip

\subsection{Summary of Results}
\label{sse:results}

\begin{enumerate}

\item \textbf{Results on the structure of the phase space:}
\begin{description}
\item{(a)} For a DAG SyDS,
the length of every phase space cycle is a power of 2.
If the number of levels of a given acyclic SyDS is $L$,
then no phase space cycle is longer than $2^L$,
and no transient is longer than $2^L-1$
(Theorem \ref{thm:levels_phase_space}).
Moreover, for each $L$,
there is an $L$-level DAG SyDS whose phase space graph is a cycle of length $2^L$
(Observation \ref{obs:long_phase_space_cycle}).

\item{(b)} For any monotone DAG-SyDS, every cycle is a fixed point, 
and the length of any transient does not exceed the number of levels
(Theorem \ref{thm:montone_stable}).

\item{(c)} For a DAG-SyDS \cals{} whose local functions are generalized-NCFs,
the length of every phase space cycle is at most 2.
Moreover, if the number of levels of  \cals{} is $L$,
no transient is longer than $2L-1$
(Theorem \ref{thm:NCF_phase_space}).
\end{description}

\item \textbf{Results for the Convergence Problem:}
\begin{description}
\item{(a)} The convergence problem for DAG SyDSs is efficiently solvable 
as shown in \cite{Chistikov-etal-2020}.
%%(Corollary \ref{cor:convergence}).

\item{(b)} The convergence guarantee problem for DAG SyDSs is co-NP-complete,
even when restricted to dags with  three levels and 3-symmetric functions
(Theorem \ref{thm:convergence_guaranee}).
\end{description}

\item \textbf{Results for the Reachability Problem:}
\begin{description}
%\item{(a)} The reachability problem is solvable in polynomial time
%for DAG SyDSs with a bounded number of levels
%(Theorem \ref{thm:reachability_bounded_levels}).

\item{(b)} The reachability problem is PSPACE-complete for DAG symmetric SyDSs
(Theorem~\ref{thm:reachability-PSPACE}).

%\item{(c)} The reachability problem is efficiently solvable 
%for DAG linear SyDSs
%(Theorem \ref{thm:linear_systems_reachability}).

\item{(d)} We define the concept of an \textbf{embedding} of one SyDS into another,
and use this concept to show that
for a fixed value of $r$, the reachability problem for directed $r$-symmetric  SyDSs 
is polynomial-time reducible to the reachability problem for directed symmetric SyDSs,
and  the reachability problem for DAG $r$-symmetric  SyDSs 
is polynomial-time reducible to the reachability problem for DAG  symmetric  SyDSs
(Corollary \ref{cor:reachability_symmetric}).
\end{description}


\iffalse
%%%%%%%%%%%%%%%%%%%%%%%%%%%
\item \textbf{Results for Garden of Eden, Fixed Point and Predecessor Existence
Problems:}

\begin{description}
\item{(a)} We define the concept of a local transition function being {\bf balanced},
and relate this concept to the Garden of Eden problem for DAG SyDSs.

\item{(b)} For fixed $r$, the GE existence problem for DAG SyDSs
whose local transition functions are specified as $r$-symmetric tables
can be solved in polynomial time
(Theorem \ref{thm:GardenEden-r_symmetric}).

\item{(c)} The GE existence problem for DAG SyDSs
whose local transition functions are specified as Boolean formulas
is NP-complete,
even when restricted to DAGs with two levels
(Theorem \ref{thm:GardenEden-NP}).

\item{(d)} The problem of determining whether a given DAG SyDS 
has a fixed point is NP-complete,
even when restricted to dags with two levels and maximum node degree 3
(Theorem \ref{thm:fixed_point}).

\item{(e)} The predecessor existence problem for DAG SyDSs  is NP-complete,
even when restricted to dags with two levels and maximum node degree 3
(Theorem \ref{thm:predecessor_existence}).
\end{description}
%%%%%%%%%%%%%%%%%%%%%%%%%%%%
\fi
\end{enumerate}

\subsection{Related Work}
\label{sse:related}

In our earlier paper on Bi-Threshold SyDSs \cite{KKM+2013}, 
we showed that for DAG bi-threshold SyDSs,
every phase space cycle is of length at most two, 
and the length of a transient is at most $2L-1$,
with the consequence that reachability can be solved in polynomial time.
We also showed that the predecessor existence problem is NP-complete, 
even when all threshold values are 1, and there are only three levels in the dag.
