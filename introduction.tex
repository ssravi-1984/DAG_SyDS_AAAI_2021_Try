\section{Introduction}
\label{sec:intro}

%\subsection{Motivation}
%\label{sse:motivation}

%\textcolor{red}{(To be revised)}

\smallskip

\textbf{Background and Motivation:}~
Discrete dynamical systems provide a formal model
for the study of diffusion phenomena in networks.
Examples of such phenomena include diffusion of social contagions
(e.g., information, opinions, fads) and epidemics
\cite{Adiga-etal-2019,Chistikov-etal-2020,OU-2020,MR-2007}.
Informally, such a dynamical system\footnote{A formal definition
is provided in Section~\ref{sec:defs}.}
consists of an underlying (social or biological) network, with each node having a
state value from a domain \bbb.
In this paper, we assume that the underlying
graph is directed and that the domain is binary (i.e., \bbb{} = \{0,1\}).
The propagation of the contagion is modeled by a collection
of Boolean local functions, one per node.
For any node $v$, the inputs to the local function $f_v$ at $v$
are the current states of $v$ and those of its in-neighbors (i.e., the nodes
from which $v$ has an incoming edges) and the output of $f_v$
is the state of $v$ at the next time instant.
We consider the \textbf{synchronous} update model, where all
nodes evaluate their local functions and update their states
\emph{in parallel}.
These dynamical systems are referred to
as \textbf{synchronous dynamical systems} (SyDSs) in
the literature (e.g., \cite{Adiga-etal-2019,Rosenkrantz-etal-2018}).
In applications involving systems biology, such systems are also
referred to as \textbf{synchronous Boolean networks}
(e.g., \cite{Kauffman-etal-2003,OU-2020,Akutsu-etal-2007}).
Throughout this paper, we will use the term SyDS to denote such a system.
The local functions may be deterministic (e.g., threshold models
used in social systems \cite{granovetter-1978}) or stochastic
(e.g., the SIR model of disease propagation \cite{Easley-Kleinberg-2010}).
We consider deterministic local functions throughout this paper.

The \textbf{configuration} of a SyDS with $n$ nodes at at time $\tau$ is a
vector $(s_1^{\tau}, s_2^{\tau}, \ldots, s_n^{\tau})$, where $s_i^{\tau}$
is the state value of node $v_i$ at time $\tau$, $1 \leq i \leq n$.
As the nodes evaluate and update their local functions, the configuration
of the system evolves over time.
If a SyDS transitions from a configuration \calc{} to a configuration \calcp{}
in one time step, we say that \calcp{} is the \textbf{successor} of \calc. 
%and that \calc{} is a \textbf{predecessor} of \calcp.
A configuration \calc{} which is its own successor is called a
\textbf{fixed point}.
Thus, once a SyDS reaches a fixed point, no further state changes
occur at any node.

When using SyDSs as models of social or biological phenomena, it is of interest 
to study whether a given SyDS starting from a specified
initial configuration  may reach certain
(desirable or undesirable) configurations.
For example, in the context of information propagation, where the
state value 1 indicates that a node has received the information
propagating through the network,
one may be interested in configurations where many nodes are in state 1.
On the other hand, in contexts such as disease propagation where
the state of 1 indicates that a node has been infected,
configurations where many nodes are in state 1 are undesirable.
One can study such \textbf{configuration reachability} problems
by considering another directed graph,
called the \textbf{phase space}, of the dynamical system \cite{MR-2007}.
Each node in the phase space is a configuration and there is an
edge from a node $X$ to node $Y$ if the SyDS can transition from
the configuration represented by $X$ to that represented by
$Y$ in one step.
Thus, the configuration reachability problem becomes a directed
path problem in phase space.
For a SyDS with $n$ nodes where the state of each node from $\{0,1\}$,
the number of possible configurations is $2^n$; thus, the size of
the phase space of a SyDS is \emph{exponential} in the size of
the SyDS.
There has been a considerable amount of research whose goal is
to understand when such reachability problems are solvable
in time that is polynomial in the size of the description of SyDS 
and when they are computationally intractable 
(see, e.g., \cite{Chistikov-etal-2020,BH+06,OU-2020,OU-2017,Akutsu-etal-2007}
and the references cited therein).

Several papers (e.g., \cite{OU-2017,Akutsu-etal-2007}) have
presented computational intractability results for reachability 
problems for SyDSs over directed graphs.
Very recently, Chistikov et al. (\citeyear{Chistikov-etal-2020})
studied two reachability problems (called Convergence and Convergence
Guarantee\footnote{These problems will be defined shortly.}) 
in the context of opinion propagation.
They show that these problems are \cpsp-complete for general
directed graphs and
that the Convergence problem can be solved
efficiently when the underlying directed graph is \emph{acylic} (i.e.,
it does not have any directed cycle) regardless of the local functions
at each node.
Following standard graph theoretic terminology, we refer to directed
acylic graphs as \textbf{dags} \cite{CLRS-2009}.
Many optimization problems (e.g., finding a longest simple directed path) which
are \cnp-hard for general directed graphs are known to be efficiently
solvable for dags \cite{GJ-1979}.
Thus, it is of interest to study the complexity of reachability problems for
SyDSs whose underlying graphs are dags.
We refer to such SyDSs as DAG-SyDSs.
Our results are summarized below.

\smallskip

\noindent
\textbf{Summary of Results and Their Significance:}

\smallskip

\noindent
\underline{(1) Results on the structure of the phase space:}
For any DAG-SyDS, we show that
the length of every phase space cycle is a power of 2.
If the number of levels in the underlying DAG of a given SyDS is $L$,
then no phase space cycle is longer than $2^L$,
and no transient (i.e., a directed path leading to a directed
cycle) is longer than $2^L-1$
(Theorem \ref{thm:levels_phase_space}).
Moreover, for each $L$, we observe that
there is an $L$-level DAG SyDS whose phase 
space graph is a cycle of length $2^L$
(Proposition \ref{pro:long_phase_space_cycle}).
As discussed in Item~(4) below,
such structural properties of phase spaces are useful in solving 
reachability problems; if the maximum lengths of transients
and cycles are bounded by polynomial functions of the size of a given SyDS,
then reachability problems for that SyDS can be solved efficiently
(by running the SyDS for a polynomial number of steps).

\smallskip

\noindent
\underline{(2) Complexity of the Reachability Problem for DAG-}\newline
\underline{SyDSs:} 
It was shown in
\cite{Chistikov-etal-2020} that the Convergence problem
(i.e., given a SyDS \cals{} and an initial configuration,
does \cals{} reach a fixed point?) is \cpsp-complete
for SyDSs on directed networks.
They also showed that the convergence problem can be solved efficiently
for DAG-SyDSs.
We show that the reachability problem
(i.e., given a SyDS \cals{} and two configurations \calc{} and \cald{},
does \cals{} starting from \calc{} reach \cald{}?),
which is similar to the Convergence problem, 
is \cpsp-complete for DAG-SyDSs even
when each local function is symmetric.
(Section~\ref{sec:defs} presents definitions of various classes
of Boolean functions.)
%It was shown in \cite{OU-2017} that the Reachability
%problem is \cpsp-complete for SyDSs on directed graphs 
%which may contain cycles.
To our knowledge, no hardness result is currently known for
the Reachability problem for DAG-SyDSs.
Our proof of this result involves two major steps.
The first step uses a reduction from the Quantified Boolean Formulas
(QBF) problem \cite{GJ-1979} to show that the Reachability problem for DAG-SyDSs is
\cpsp-complete even when each local function is 
$r$-symmetric\footnote{The definition of $r$-symmetric functions is given in
Section~\ref{sec:defs}. Symmetric Boolean functions
are 1-symmetric. As will be explained in Section~\ref{sec:defs},
the local functions used in \cite{Chistikov-etal-2020} are
2-symmetric.}
for some constant $r$ (Theorem~\ref{thm:reach-six-sym}).
For the second step,
we define the concept of an \textbf{embedding} of one SyDS into another,
and use this concept to show that
for any fixed value of $r$, 
the reachability problem for DAG-SyDSs with $r$-symmetric local functions 
is polynomial-time reducible to the Reachability problem for DAG-SyDSs with
symmetric  local functions (Theorem \ref{thm:reachability_symmetric}).
The \cpsp-completeness of the Reachability problem for DAG-SyDSs with 
symmetric local function (Theorem~\ref{thm:dag_syds_sym_hard})
follows directly from Theorems~\ref{thm:reach-six-sym} and 
\ref{thm:reachability_symmetric}.

\smallskip
The difference between the complexities of Convergence
and Reachability problems for DAG-SyDSs is due to the following.
For any DAG-SyDS with $L$ levels, regardless of the local functions,
the length of any transient leading to a fixed point is bounded
by $L$. (This is a slight restatement of Proposition~1 
in \cite{Chistikov-etal-2020}.)
Thus, the Convergence problem for DAG-SyDSs is efficiently solvable.
On the other hand, we show that one can construct DAG-SyDSs with specific local
functions such that the length of a transient and that of 
the cycle the transient leads to are both exponential in $L$
(Theorem~\ref{thm:path_length_lower_bounds}).
The idea behind this construction enables us 
to establish \cpsp-hardness result for
the reachability problem for DAG-SyDSs.

\smallskip

\noindent
\underline{(3) Complexity of the Convergence Guarantee Problem:}
It was shown in \cite{Chistikov-etal-2020} that
the Convergence Guarantee problem 
(i.e., given a SyDS \cals{} on
a directed graph, does \cals{} reach a fixed point from every
initial configuration?)  
is \cpsp-complete for SyDSs on general directed graphs.
For DAG-SyDSs, we show that the problem is \cconp-complete,
even when restricted to dags with at most  three levels 
(Theorem \ref{thm:convergence_guaranee}).
Thus, our result points out that the problem remains 
computationally intractable even for DAG-SyDSs.

\smallskip

\noindent
\underline{(4) Efficient Solvability of Reachability Problem for Special}\newline
\underline{Classes of DAG-SyDSs:} 
We show that the Reachability problem is efficiently solvable for
DAG-SyDSs whose local functions are from two specific
classes of Boolean functions, namely monotone functions and
nested canalyzing functions (NCFs).
For DAG-SyDSs with monotone local functions, we show that every cycle in the
phase space is a fixed point (Theorem \ref{thm:monotone_stable}).
For DAG-SyDSs with NCF local functions, we show that each phase 
space cycle is of length at most 2
and that the maximum length of a transient is $2L-1$, where $L$
is the number of levels in the dag
(Theorem~\ref{thm:NCF_phase_space} in the supplement).
These properties lead to
the efficient solvability of the reachability
problem for these two cases.

\iffalse
%%%%%%%%%%%%%%%%%%%%
The results in \cite{Rosenkrantz-etal-2018} readily imply
that the Reachability problem for SyDSs whose local
functions are NCFs and whose directed graphs may contain cycles  
is \cpsp-complete.
To our knowledge, the only previous result on the efficient
solvability of the  Reachability problem for DAG-SyDSs is
for the case when each local function is a bithreshold
function \cite{KKM+2013}.

 we show
that the length of every phase space cycle is at most 2.
Moreover, if the number of levels of  \cals{} is $L$,
no transient is longer than $2L-1$
(Theorem \ref{thm:NCF_phase_space}).
These properties
directly imply the efficient solvability of the reachability
problem for such DAG-SyDSs.
%%%%%%%%%%%%%%%%%%%%
\fi


%\subsection{Related Work}
%\label{sse:related}

\smallskip

\noindent
\textbf{Related Work:}~ Reachability problems for various
models of discrete dynamical systems have been widely studied
in the AI literature.
For example, complexity results for reachability and related problems 
for Hopfield neural nets were studied in 
\cite{FO-1989, Orponen-1993,Orponen-1994}. 
Problems related to the diffusion of opinions and other contagions have
also been studied under various discrete dynamical systems
models (see e.g., 
\cite{Auletta-etal-2018,Botan-etal-2019,Chistikov-etal-2020,BE-2017}
and the references contained therein).

Motivated by dynamical system models for large-scale simulations,
Barrett et al. (\citeyear{BH+06}) studied the reachability problem
for discrete dynamical systems where the underlying graph is 
undirected and the update model is sequential (i.e., nodes compute 
and update their state values in a specified order). 
Synchronous dynamical systems over 
directed graphs serve as useful models for biological
phenomena \cite{Kauffman-etal-2003}.
Many researchers have studied reachability and related 
problems for such dynamical systems (see e.g., 
\cite{OU-2020,OU-2017,Akutsu-etal-2007} and
the references cited therein).
The results in \cite{Rosenkrantz-etal-2018} readily imply
that the Reachability problem for SyDSs whose local
functions are NCFs and whose directed graphs may contain cycles
is \cpsp-complete.
To our knowledge, the reachability problem for DAG-SyDSs 
was first considered in \cite{KKM+2013}. 
They showed that when each local function is 
a bi-threshold function (i.e., each node has two threshold
values to control the $0 ~\rightarrow~  1$ and $1 \rightarrow 0$
transitions), the problem can be solved efficiently.
%By establishing bounds on the lengths of cycles and transients,
%they showed that the reachability problem can be efficiently
%solved for such DAG-SyDSs.

