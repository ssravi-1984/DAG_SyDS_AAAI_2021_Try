\subsection{DAG-SyDSs with Monotone Local Functions}
\label{sse:monotoner_functions}

\begin{theorem}\label{thm:montone_stable}
For a DAG monotone SyDS, every cycle is a fixed point, 
and the length of any transient does not exceed the number of levels.
\end{theorem}
\noindent
\textbf{Proof:}~ 
By induction on the number of levels for each level $i$, 
each level $i$ node is stable after at most $i$ steps.

From Lemma \ref{lem:level_zero_nodes}, every level 0 node is either stable or alternating  at time 1.
The complement function is not monotone, so a  level 0 node is cannot be alternating,
and thus is stable at time 1.

Suppose that all the incoming edges to a given node $v$ have stable values at time $t$.
From Lemma \ref{lem:all_inputs_stable}, node $v$ is either stable or alternating  at time $t+1$.
Since the local transition function for node $v$ is monotone, 
and all the incoming edges have stable values at time $t$, node $v$ is stable at time $t+1$
\QED

We observe that that for every $L \geq 0$, there exists a DAG monotone SyDS with 
a transient of length $L$. 
Consider the SyDS whose underlying graph is a directed chain of $L$ nodes.
The local transition function of the level 0 node is the constant 1,
and of every other node is the {\em or} of its value and the value from the incoming edge.
The configuration of all zeros takes $L$ steps to reach the fixed point of all ones.
