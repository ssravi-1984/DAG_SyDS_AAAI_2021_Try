\section{Definitions and Notation}
\label{sec:defs}

\smallskip
\noindent
\underline{\textsf{SyDSs on directed graphs:}}
Let \bbb{} denote the Boolean domain \{0,1\}.
In this paper, a \textbf{Synchronous Dynamical System} (SyDS)~
\cals{} over \bbb{} is assumed to be specified as
a pair ${\cals}  = (G, {\calf})$, where
(i) $G(V,E)$, a directed graph with $|V| = n$,
represents the underlying graph of the SyDS,
with node set~$V$ and (directed) edge set~$E$, and
(ii) ${\calf} = \{f_1, f_2, \ldots, f_n \}$ is a collection of
functions in the system, with
$f_i$ denoting the \textbf{local transition
function} associated with node $v_i$, $1 \leq i \leq n$.
Each node of $G$ has a state value from \bbb.
Each Boolean function $f_i$ specifies the local interaction
between node $v_i$ and its in-neighbors in $G$.
The inputs to function $f_i$ are the state of $v_i$ and
those of the in-neighbors of $v_i$ in $G$;
function $f_i$ maps
each combination of inputs to a value in \bbb.
This value becomes the next state of node $v_i$.
It is assumed that each local function can be evaluated in
polynomial time.
In a SyDS, all nodes compute and update their next state
\emph{synchronously}.
Other update disciplines (e.g., sequential updates)
have also been considered in the literature (e.g., \cite{MR-2007}).
At any time $\tau$,
the {\bf configuration} \calc{} of a SyDS
is the $n$-vector $(s_1^{\tau}, s_2^{\tau}, \ldots, s_n^{\tau})$,
where $s_i^{\tau} \in \bbb$ is the state of
node $v_i$ at time $\tau$ ($1 \leq i \leq n$).
A \textbf{DAG-SyDS} is a SyDS whose underlying graph is a dag.

\begin{figure}
\begin{center}
\input{dag_syds_example.pdf_t}
\end{center}
\caption{An example of a DAG-SyDS.
The local functions at nodes $v_1$ through
$v_6$ are Identity, Complement, Identity, AND, OR and
XOR respectively.}
\label{fig:ex_dag_syds}
\end{figure}

\smallskip

\noindent
\textbf{Example:}~ An example of a DAG-SyDS is shown in
Figure~\ref{fig:ex_dag_syds}.
Here, the local functions at nodes $v_1$ through
$v_6$ are Identity, Complement, Identity, AND, OR and
XOR respectively.
(The Identity and Complement functions have only one input;
they return respectively the input value and its complement.) 
Suppose the initial configuration of the SyDS is $(1, 0, 1, 0, 0, 0)$;
that is, at time $\tau = 0$, $v_1$ is in state 1, 
$v_2$ is in state 0, $\ldots$, $v_6$ is in state 0.
Recall that the inputs to the local function at a node $v$ are the current
state of $v$ and those of its in-neighbors (if any).
Since the local functions at $v_1$ and $v_3$ are Identity functions,
the states of $v_1$ and $v_3$ will remain 1 in all subsequent time steps.
Since the local function at $v_2$ is the Complement function, the state of
$v_2$ at time $\tau = 1$ is 1.
Using the local functions at $v_4$, $v_5$ and $v_6$, one can verify that
their states at time $\tau = 1$ are 0, 1 and 0 respectively.
Thus, the configuration at time $\tau = 1$ is 
$(1, 1, 1, 0, 1, 0)$.
Similarly, the configuration at times $\tau = 2$ and
$\tau = 3$ are  $(1, 0, 1, 0, 1, 1)$ and $(1, 1, 1, 0, 1, 0)$
respectively.
Thus, the configuration at time $\tau = 3$ is the same as that
at $\tau = 1$.
In other words, the SyDS cycles between the two configurations
at times $\tau =1$ and $\tau = 2$.

\smallskip
\noindent
\underline{\textsf{Additional definitions regarding SyDSs:}}~
Some additional concepts
related to SyDSs are defined below.

\begin{definition}\label{def:variable_typesl}
For the local transition function $f_v$ of a given node $v$ of a directed  SyDS,
we call the variable corresponding to the source node 
of each incoming edge to $v$ an {\bf incoming variable} of $f_v$,
and call the variable corresponding to $v$ the  {\bf self variable} of $f_v$.
\end{definition}

The \textbf{phase space} of a SyDS was defined in Section~\ref{sec:intro}.
A \textbf{transient} in phase space is a simple directed path $P$ whose
last node is part of a directed cycle which does not 
contain any other node of $P$.
The length of a phase space cycle is the number of edges in the cycle,
and the length of a transient is the number of edges in the transient.

Let \cals{} be a SyDS.
For a given configuration \calc{} and node $v$,
we let $\calc(v)$ denote the state of node $v$ in \calc{}.
For a given configuration \calc{} and set of nodes $Y$,
we let $\calc[Y]$ denote the projection of \calc{} onto $Y$.
We assume that the initial configuration of the system 
occurs at time 0.
For a given initial configuration \calc{} and nonnegative integer $i$,
we let $\calc_i$ denote the configuration of \cals{} at time $i$.

For a given initial configuration \calc{},
we  say that a given node $v$ is \textbf{stable} at time $t$ if
for all $i \geq t$, $\calc_i(v)= \calc_t(v)$.
Also, we say that a given node $v$ is \textbf{alternating} at time $t$ if
for all $i \geq 0$, 
$\calc_{t+2i}(v) =  \calc_t(v)$ 
and $\calc_{t+2i+1}(v) = \overline{ \calc_t(v)}$ 
(i.e., the state of $v$ alternates between 0 and 1).  
%The following lemma shows an important property of some
%nodes of a \btsyds{} whose underlying graph is a dag.


\smallskip
\noindent
\underline{\textsf{Graph theoretic definitions:}}~
Given a dag, we say that node $u$ {\bf directly precedes}
node $v$ if the graph contains an edge from $u$ to $v$,
that $u$ {\bf precedes} $v$ if the graph contains a path (possibly
with no edges) from $u$ to $v$,
and that $u$ {\bf properly precedes} $v$
if the graph contains a path with at least one edge  from $u$ to $v$.
Note that the local transition function for a given node $v$  in a
directed SyDS is a function of $v$
and the nodes that directly precede $v$.

\begin{definition}\label{def:dag_level}
For a dag $G(V,E)$, 
the \textbf{level} of a node $v \in V$ is the maximum number
of edges in any directed path to $v$.
\end{definition}

Suppose a given dag $G(V,E)$ has $L$ levels. 
For each $j$, $0 \leq j < L$, we let $\call_j$ denote the set of nodes at level $j$,
and let $\call_j'$ denote the nodes whose level is at most $j$.

\smallskip
\noindent
\underline{\textsf{Boolean function definitions:}}~
Here, we provide definitions of several classes of
Boolean functions used in this paper.
These definitions are from \cite{Crama-Hammer-2011,BH+06}.  

Consider assignments $\alpha$ and $\beta$
to a set of Boolean variables $X$.  We say that $\alpha \leq \beta$
if for every variable $x \in X$, $\alpha(x) \leq \beta(x)$.  A
Boolean function $f$  is {\bf monotone nondecreasing} if for every
pair of assignments $\alpha$ and $\beta$ to its variables, $\alpha
\leq \beta$ implies that $f(\alpha) \leq f(\beta)$, and is {\bf
monotone nonincreasing} if for every pair of assignments $\alpha$
and $\beta$ to its variables, $\alpha \leq \beta$ implies that
$f(\alpha) \geq f(\beta)$.  We say that $f$ is {\bf monotone} if
$f$ is either monotone nondecreasing or monotone nonincreasing. 
Common examples of monotone functions are OR and AND.
A {\bf monotone SyDS} is a SyDS whose local transition functions are
all monotone.

A \textbf{symmetric} Boolean function is one whose
value does not depend on the order in
which the input bits are specified;
that is, the function value depends only on how many
of its inputs are 1.
For example, the XOR function is symmetric since the 
value of the function is determined by the parity of the number 
of 1's in the input.

In symmetric Boolean functions, we need not know whether a specific
input is 0 or 1; it is enough to know the number of 1s in the input.
A similar situation arises in anonymous symmetric pure strategy
graphical games where each player's strategy is from \{0,1\}. 
A player $X$ doesn't know whether a specific neighbor chose 0 or 1; 
$X$ knows how many neighbors chose 1.
Convergence in SyDSs corresponds to reaching
a fixed point where everyone is satisfied with the outcome. 
This is similar to a Nash equilibrium in the game theoretic setting.
%Symmetric SyDSs
%can be thought of as anonymous pure strategy
%games with two strategies.  
In this sense, the convergence
problem is related to the problem of Nash equilibria in such games and
reachability is useful in understanding whether players can achieve
certain equilibria (see e.g.,
~\cite{blonski1999anonymous,daskalakis2007computing,daskalakis2015approximate,jackson2010social}).

A Boolean function $f$ is \textbf{$r$-symmetric}
if the inputs to $f$ can be partitioned into at most $r$ classes
such that the value of $f$ depends only on how many of the
inputs in each of the $r$ classes are 1.
Note that symmetric functions defined above are 1-symmetric.
For each node $v$, the local (majority) function 
used in \cite{Chistikov-etal-2020}
to study opinion diffusion can be seen to be 2-symmetric:
one class contains just the node $v$ and the other class
contains all the in-neighbors of $v$.
The value of the function is determined by the number
of 1's in these two classes.

A SyDS is $r$-symmetric
if each of its local transition
functions is $r'$-symmetric for some $r' \leq r$.

\smallskip
\noindent
\underline{\textsf{Problem definitions:}}~ We now provide
formal specifications of the problems considered in this paper.
%\label{sse:prob_def}

\smallskip
\noindent
\textbf{Reachability} %(\textsc{Reach})

\smallskip
\noindent
\underline{\textsf{Instance:}}~ A SyDS \cals{} specified 
by an underlying directed
graph $G(V,E)$ and a local transition function $f_v$ for each node $v \in V$
and two configurations \calc{} and \cald{}. 

%\smallskip
\noindent
\underline{\textsf{Question:}}~ Starting from configuration \calc,
does \cals{} reach configuration \cald? 

\iffalse
%%%%%%%%%%%%%%%%%%%%%%%
We observe that \textsc{Reach} is solvable in polynomial time for
any class of SyDSs for which function evaluation can be done in
polynomial time and there is a polynomial bound on the length of a
phase space cycle and the length of a transient.  In such a case,
the given SyDS can be simulated for the required number of steps.
As will be seen, this approach is applicable to NCFs, monotone
functions (with both positive and negative monotone functions allowed
in \cals{}), and bithreshold functions.
%%%%%%%%%%%%%%%%%%%%%%%
\fi

\smallskip
\noindent
\textbf{Convergence} %(\textsc{Convergence})

\smallskip
\noindent
\underline{\textsf{Instance:}}~ A SyDS \cals{} specified 
by a directed graph $G(V,E)$, a local function $f_v$ 
for each $v \in V$ and a configuration \calc{}. 

%\smallskip
\noindent
\underline{\textsf{Question:}}~ Starting from configuration \calc,
does  \cals{} reach a fixed point? 


%Similar to \textsc{Reach}, \textsc{Convergence} is
%solvable in polynomial time for any class of SyDSs for which local function
%evaluation can be done in polynomial time and there is a polynomial
%bound on the length of a phase space cycle and the length of a
%transient.  
%In such a case, the given SyDS can be simulated for the
%required number of steps.


\smallskip
\noindent
\textbf{Convergence Guarantee} %(\textsc{Convergence Guarantee})

\smallskip
\noindent
\underline{\textsf{Instance:}}~ A SyDS \cals{} specified 
by a directed
graph $G(V,E)$ and a local function $f_v$ for each $v \in V$. 

%\smallskip
\noindent
\underline{\textsf{Question:}}~ Does \cals{} reach a fixed point from
\emph{every} initial configuration?

The formulation of the Convergence Guarantee problem 
in \cite{Chistikov-etal-2020} considers the negation of the above
question (i.e., is there a configuration \calc{} from which
the SyDS does not reach a fixed point?).
However, since the complexity class \cpsp{} is closed under
complement \cite{Papadimitriou-1993}, there is no difference in
the complexity of the two versions of the problem.
