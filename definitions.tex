\section{ Definitions and Notation}
\label{sec:defs}

Given a directed graph, we say that node $u$ {\bf directly precedes} node $v$ 
if the graph contains an edge from $u$ to $v$,
 that $u$ {\bf precedes} $v$ if the graph contains a path (possibly with no edges) from $u$ to $v$,
and that $u$ {\bf properly precedes} $v$
 if the graph contains a path with at least one edge  from $u$ to $v$.
Note that the local transition function for a given node $v$  in a directed SyDS is a function of $v$
 and the nodes that directly precede $v$.

\begin{definition}\label{def:variable_typesl}
For the local transition function $f_v$ of a given node $v$ of a directed  SyDS,
we call the variable corresponding to the source node 
of each incoming edge to $v$ an {\bf incoming variable} of $f_v$,
and call the variable corresponding to $v$ the  {\bf self variable} of $f_v$.
\end{definition}

\begin{definition}\label{def:dag_level}
Let $G(V,E)$ be a dag.
The \textbf{level} of a node $v$  in $G$ is the maximum number
of edges in any directed path to $v$.
\end{definition}

Suppose a given dag $G(V,E)$ has $L$ levels. 
For each $j$, $0 \leq j < L$, we let $\call_j$ denote the set of nodes at level $j$,
and let $\call_j'$ denote the nodes whose level is at most $j$.

The {\bf length} of a phase space cycle is the number of edges in the cycle,
and the {\bf length} of a phase space transient is the number of edges in the transient.

Let \cals{} be a SyDS.
For a given configuration \calc{} and node $v$,
we let $\calc(v)$ denote the state of node $v$ in \calc{}.
For a given configuration \calc{} and set of nodes $Y$,
we let $\calc[Y]$ denote the projection of \calc{} onto $Y$.
We assume that the initial configuration of the system 
occurs at time 0.
For a given initial configuration \calc{} and nonnegative integer $i$,
we let $\calc_i$ denote the configuration of \cals{} at time $i$.

For a given initial configuration \calc{},
we  say that a given node $v$ is \textbf{stable} at time $t$ if
for all $i \geq t$, $\calc_i(v)= \calc_t(v)$.
Also, we say that a given node $v$ is \textbf{alternating} at time $t$ if
for all $i \geq 0$, 
$\calc_{t+2i}(v) =  \calc_t(v)$ 
and $\calc_{t+2i+1}(v) = \overline{ \calc_t(v)}$ 
(i.e., the state of $v$ alternates between 0 and 1).  
%The following lemma shows an important property of some
%nodes of a \btsyds{} whose underlying graph is a dag.

A {\bf linear} Boolean function is one that can be expressed as a linear equation mod 2 of its variables,
i.e., the xor or complement of the xor of its variables.
A {\bf linear SyDS} is a SyDS whose local transition functions are all linear.

Consider assignments $\alpha$ and $\beta$ to a set of Boolean variables $X$.
We say that $\alpha \leq \beta$ if for every variable $x \in X$, $\alpha(x) \leq \beta(x)$.
A Boolean function $f$  is {\bf monotone nondecreasing}
if for every pair of assignments $\alpha$ and $\beta$ to its variables,
$\alpha \leq \beta$ implies that $f(\alpha) \leq f(\beta)$,
and is {\bf monotone nonincreasing}
if for every pair of assignments $\alpha$ and $\beta$ to its variables,
$\alpha \leq \beta$ implies that $f(\alpha) \geq f(\beta)$.
We say that $f$ is {\bf monotone} if $f$ is either monotone nondecreasing or monotone nonincreasing
A {\bf monotone SyDS} is a SyDS whose local transition functions are all monotone.


\iffalse
%%%%%%%%%%%%%%%%%%%%%%%%%%%%%%%%%%%%%%%
\medskip
\noindent
\textbf{Predecessor Existence} (\pre)

\smallskip
\noindent
\underline{\textsf{Instance:}}~ A SyDS \cals{} specified 
by an underlying
graph $G(V,E)$ and a local transition function $f_v$ for each node $v \in V$;
a configuration \calc{} for \cals.

\smallskip
\noindent
\underline{\textsf{Question:}}~ Does \calc{} have a predecessor, 
that is, is there
a configuration \calcp{} such that \cals{} has a one step transition
from \calcp{} to \calc?  

\medskip
The corresponding counting problem (i.e., finding the number
of predecessors of \calc) will be denoted by \npre.

\medskip
\noindent
\textbf{Fixed Point Existence} (\textsc{Fixed Point})

\smallskip
\noindent
\underline{\textsf{Instance:}}~ A SyDS \cals{} specified 
by an underlying
graph $G(V,E)$ and a local transition function $f_v$ for each node $v \in V$.

\smallskip
\noindent
\underline{\textsf{Question:}}~ Does \cals{} have a fixed point?  
%%%%%%%%%%%%%%%%%%%%%%%%%%%%%%%%%%%%%%%
\fi

\subsection{Problem Definitions}
\label{sse:prob_def}

\smallskip
\noindent
\textbf{Configuration Reachability} (\textsc{Reach})

\smallskip
\noindent
\underline{\textsf{Instance:}}~ A SyDS \cals{} specified 
by an underlying
graph $G(V,E)$ and a local transition function $f_v$ for each node $v \in V$,
and two configurations \calc{} and \cald{}. 

%\smallskip
\noindent
\underline{\textsf{Question:}}~ Starting from configuration \calc,
does  \cals{} reach configuration \cald? 

We observe that \textsc{Reach} is solvable in polynomial time for
any class of SyDSs for which function evaluation can be done in
polynomial time and there is a polynomial bound on the length of a
phase space cycle and the length of a transient.  In such a case,
the given SyDS can be simulated for the required number of steps.
As will be seen, this approach is applicable to NCFs, monotone
functions (with both positive and negative monotone functions allowed
in \cals{}), and bithreshold functions.

\smallskip
\noindent
\textbf{Convergence} (\textsc{Convergence})

\smallskip
\noindent
\underline{\textsf{Instance:}}~ A SyDS \cals{} specified 
by an underlying
graph $G(V,E)$ and a local transition function $f_v$ for each node $v \in V$,
and a configuration \calc{}. 

%\smallskip
\noindent
\underline{\textsf{Question:}}~ Starting from configuration \calc,
does  \cals{} reach a fixed point? 

We observe that similar to \textsc{Reach}, \textsc{Convergence} is
solvable in polynomial time for any class of SyDSs for which function
evaluation can be done in polynomial time and there is a polynomial
bound on the length of a phase space cycle and the length of a
transient.  In such a case, the given SyDS can be simulated for the
required number of steps.


\smallskip
\noindent
\textbf{Convergence Guarantee} (\textsc{Convergence Guarantee})

\smallskip
\noindent
\underline{\textsf{Instance:}}~ A SyDS \cals{} specified 
by an underlying
graph $G(V,E)$ and a local transition function $f_v$ for each node $v \in V$. 

%\smallskip
\noindent
\underline{\textsf{Question:}}~ Does  \cals{} always reach a fixed point,
that is, does the phase space of  \cals{} contain a cycle with more than one node? 

