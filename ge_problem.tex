\section{Garden of Eden Problem}
\label{sec:garden_eden}

We now consider the GE problem.
\smallskip
\begin{definition}\label{def:balanced_function}
For a node $v$ of a directed SyDS, an {\bf incoming assignment} is an assignment of a value to 
each of the variables corresponding to the source node of an incoming edge.
For a Boolean value $a$ and incoming assignment $\alpha$,
we use the notation $f_v(a,\alpha)$ to denote the value of 
local transition function $f_v$ when $x = a$ and the other variables have the values in $\alpha$.

A local transition function $f_v$ of a node $v$ of a given directed SyDS
is {\bf balanced} if for every incoming assignment $\alpha$, $f_v(0,\alpha) \neq f_v(1,\alpha)$.
\end{definition}

\smallskip
\begin{theorem}\label{thm:GardenEden-balanced}
A DAG SyDS \cals{} has a Garden of Eden configuration iff
at least one of its local transition functions is unbalanced.
\end{theorem}
\noindent
\textbf{Proof:}~
{\bf If:} Suppose that \cals{} has at least one unbalanced local transition function.
Let $v$ be a node such that its local transition function is unbalanced,
but all its proper predecessors have balanced local transition functions.
Let $Y$ denote all the predecessors of $v$
(including $v$ itself.)
Then there exist two configurations \calc{} and \cald{} of $Y$,
such that they differ only in the value of $v$,
but have the same successor configuration on $Y$.
Since there are two configurations of $Y$ with the same successor,
there is at least one configuration of $Y$ with no predecessor configuration of $Y$.
Any extension of this configuration to a full configuration of \cals{} 
has no predecessor, and so is a GE configuration of \cals{}.

\noindent
{\bf Only If:}
Suppose that all local transition functions of  \cals{} are balanced.
We now argue that every pair of distinct configurations of  \cals{} has distinct successors.
Let  \calc{} and \cald{} be distinct configurations of \cals{}.
Let $j$ be the lowest level number of \cals{}
such that  \calc{} and \cald{} have the same values on all nodes on lower levels.
Then there is at least one node $v$ in level $j$ such that $\calc[v] \neq \cald[v]$.
However, the incoming assignment to $v$ is the same in \calc{} and \cald{}.
Since the local transition function of node $v$ is balanced,
$\calc_1[v] \neq \cald_1[v]$.
Thus every pair of distinct configurations of  \cals{} has distinct successors.
Consequently,  \cals{} has no GE Configurations.
\QED

\smallskip
\begin{theorem}\label{thm:GardenEden-r_symmetric}
For fixed $r$, the GE existence problem for DAG SyDSs
whose local transition functions are specified as $r$-symmetric tables
can be solved in polynomial time.
\end{theorem}
\noindent
\textbf{Proof:}~
Each table can be tested for the property of being balanced.
The check can be done  in time proportional to the size of the table.
\QED

\smallskip
\begin{theorem}\label{thm:GardenEden-NP}
The GE existence problem for DAG SyDSs
whose local transition functions are specified as Boolean formulas
is NP-complete,
even when restricted to DAGs with two levels.
\end{theorem}
\noindent
\textbf{Proof:}~The proof is via a reduction from 3SAT.
Let $f$ denote the given 3SAT Boolean formula.
Let $n$ be the number of variables in $f$.
For the reduction, we construct a SyDS \cals{} with $n+1$ nodes.
Let $X$ denote the first $n$ nodes of \cals{}, all of which are in layer zero.
The local transition function of each of these nodes is the identity function.
Node $x_{n+1}$ has an incoming edge from each of the other $n$  nodes.
Its local transition function is:  $x_{n+1} \vee f$.

Note that the local transition function for each node in $X$ is balanced.
Also note that every assignment to the nodes in $X$ 
is an incoming assignment to node $x_{n+1}$.
%Finally, note that the local transition function for $x_{n+1}$ can  change the value of $x_{n+1}$ 
%only if $x_{n+1} = 0$ and the values of the other variables satisfy $f$.

Suppose that formula $f$ is satisfiable. 
Let $\alpha$ be a satisfying assignment for $f$.
Then, $f_{x_{n+1}}(0,\alpha) = f_{x_{n+1}}(1,\alpha) = 1$, so $f_{x_{n+1}}$ is not balanced.

Now suppose that formula $f$ is unsatisfiable. 
Then, the local transition function for $x_{n+1}$ is the identity function, which is balanced.

Thus, formula $f$ is satisfiable iff the local transition function for $x_{n+1}$ is unbalanced.
Consequently, from Theorem \ref{thm:GardenEden-balanced},
 $f$ is satisfiable iff  \cals{} has a Garden of Eden configuration.
So, GE existence is NP-hard.

We note that the GE existence problem is in NP for all directed or undirected SyDSs,
since a GE configuration exists iff there are two distinct configurations with a common successor.
\QED
