\section{Convergence Guarantee for DAG-SyDSs}
\label{sec:conv_guarantee}

As mentioned earlier, \cite{Chistikov-etal-2020}
showwed that the convergence problem and the convergence
guarantee problem are \cpsp-complete for SyDSs on general directed
graphs.
They also showed that the convergence problem is efficiently
solvable for DAG SyDSs.
However, they did not address the convergence guarantee problem
for DAG SyDSs.
In this section, we show that this problem 
remains computationally intractable even for
DAG SyDSs with three levels.

\begin{theorem}\label{thm:convergence_guaranee}
The convergence guarantee problem for DAG SyDSs is co-NP-complete,
even when restricted to dags with three levels and 3-symmetric functions.
\end{theorem}

\noindent
\textbf{Proof:}~ The problem is in \cconp{} since when the answer to the
convergence guarantee is ``NO", one can guess a specific inital configuration,
run the DAG SyDS for a number of steps equal to the number of levels, and
thus demonstrate efficiently that the system has not reached a fixed point.
To prove \cnp-hardness, we use a reduction from 3SAT.
% similar to that used in the proof of Theorem \ref{thm:fixed_point}.
Given an instance of 3SAT,
the constructed DAG-SyDS $\cals$ has a level zero node for each variable, 
a level one node for each clause, and a single node at level two.
We denote these three sets of nodes as $\{x_1, x_2, \ldots , x_n\}$,
$\{c_1,c_2, \ldots , c_m\}$, and $\{y\}$, respectively.

There is a directed edge from a  level zero node for a given variable
to each of  the nodes for the clauses involving that variable.
There is a directed edge from each level one node to node $y$.  The
local transition function for each level zero node is the identity
function.  The local transition function for each level one node
is the function that equals 1
if the value of at least one of the incoming edges
makes the clause true.  (This transition function is 3-symmetric
because at least two of the literals are positive or negative.) The
local transition function for $y$ is the function 
$\bar{y} \, \wedge \, c_1 c_2 \cdots  c_{m}$.

Note that for every initial configuration, the level zero nodes are
all stable at time 0, and the level one nodes are all stable at
time 1.

%\smallskip
Suppose the given 3SAT problem instance is satisfiable.
Then any initial configuration of $\cals$ where the level zero nodes 
have values corresponding to a satisfying assignment
results in node $y$ being alternating at time 1.
Suppose the given 3SAT problem instance is unsatisfiable.
Then for every initial configuration, node $y$ is stable with value 0 at time 2.

\smallskip
Thus, $\cals$ reaches a fixed point from every initial configuration, 
i.e.,  convergence is guaranteed for $\cals$,
iff the given 3SAT problem instance is unsatisfiable.
The theorem follows.
\QED
