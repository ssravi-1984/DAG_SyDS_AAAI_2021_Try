\section{Reachability Problem for DAG-SyDSs}
\label{sec:reachability}

%\smallskip
\begin{theorem}\label{thm:reachability-PSPACE}
The reachability problem is \cpsp-complete for DAG-SyDSs
where each local function is symmetric.
\end{theorem}

\noindent
\textbf{Proof:}~ 
The proof is via a reduction from Quantified Boolean Formulas (QBF).
Let $F$ denote the given quantified Boolean formula.
Let $f$ denote the Boolean expression that is quantified.
Without loss of generality, we can assume that $f$ is a 3SAT formula.
Let $X = \{x_0, x_1, \dots , x_{n-1}\}$ be the set of variables of $f$,
and $\{c_0, c_1, \dots , c_{m-1}\}$ be the set of clauses of $f$.
Let $F$ be $(Q_ {n-1} x_{n-1}) \cdots (Q_1 x_1) (Q_0 x_0) f$,
where each $Q_i$ is either $\forall$ or $\exists$.       

We first give a reduction where the constructed DAG SyDS is 6-symmetric.

For each $i$, $0 \leq i <n$,
 we use the notation $o_i$ to denote a binary Boolean operation as follows.
 If $Q_i$ is $\forall$, then $o_i$ is {\em AND};
 if $Q_i$ is $\exists$, then $o_i$ is {\em OR}.

For the reduction, we construct a reachability problem instance whose SyDS \cals{} 
has an underlying graph with $2n+m+2$ nodes, on $2n+2$ layers. 

SyDS \cals{} contains the following nodes.
We let $Y = \{y_0, y_1, \dots , y_n\}$ be a set of $n+1$ nodes,
$R = \{r_0, r_1, \dots , r_{n-1}\}$ be a set of $n$ nodes,
$W = \{w_0, w_1, \dots , w_{m-1}\}$ be a set of $m$ nodes,
 and $h$ be an extra node.
 
The initial configuration \calc{} for the constructed problem instance
 has all nodes equal to 0.

The goal configuration \cald{} for the constructed problem instance 
has $h =1$, $r_{n-1} = 1$,  each  $r_i= 0$ for $0 \leq i  < n-1$, 
each  $y_i= 0$, 
and each $w_j$ equal to 1 iff the corresponding clause $c_j$ contains a positive literal.

Node set $Y$ will function as a cyclical counter,
with incoming edges and local transition functions similar to those for
the $n$ nodes in the construction of Proposition \ref{pro:long_phase_space_cycle}.
For each $i$, $0 \leq i < n$,
node $y_i$ will correspond to the variable $x_i$ in $f$.

Each node $w_i \in W$ will correspond to clause $c_i$.

Each node $r_i \in R$ will correspond to the quantified Boolean formula
$(Q_i x_i) \cdots  (Q_0 x_0) f$.
In particular, at some point in the operation of \cals{},
node $r_{n-1}$ will have the value of the quantified Boolean formula $F$.

Node $h$ will serve as a control node.
Once the value of node $h$ equals 1,  it remains 1 forever more,
forces node $r_{n-1}$ to retain its value, and forces all the other nodes in $R$ to have value 0.

The underlying graph of \cals{} has directed edges
$(y_i, y_j)$ for each $i$ and $j$ such that $0 \leq i < j \leq n$.
For each node $w_j$, there are incoming edges from those $Y$ nodes
corresponding to the variables occurring in clause $c_j$.
Node $h$ has incoming edges from the $n+1$ nodes in $Y$.
Node $r_0$ has an incoming edge from $y_0$, $h$, and from each node in $W$.
For each node $r_i$, $1 \leq i < n$,
there are incoming edges from $r_{i-1}$, $h$, and each $y_j$ where $0 \leq j \leq i$.


The local transition function for each node in $Y$ is the function 
that retains the current value of the node if any of the incoming variables equals 0,
and changes the value of the node otherwise.

The local transition function for each node $w_i \in W$ is the same as clause $c_i$,
but using the values of the incoming variables.

The local transition function for $h$ is as follows.
If $h=1$, then 1.
If $h =0$ and  the $n+1$ incoming edges from $Y$ encode the integer $2^n+n-1$, then 1.
Otherwise, $0$.

The local transition function for $r_0$ is as follows.
If $h =1$, then 0.
Otherwise, if $y_0 = 1$, then the {\em and} of the incoming variables from $W$.
Otherwise, operation $o_0$ 
applied to $r_0$ and the {\em and} of the incoming variables from $W$.

The local transition function for $r_i$ , $1 \leq i < n-1$, is as follows.
If $h =1$, then 0.
Otherwise, if the $i+1$ incoming edges from $Y$ encode the integer $2^i+i$, 
then $r_{i-1}$.
Otherwise,  if the incoming edges from $Y$ encode the integer $i$, 
then operation $o_0$ applied to $r_i$ and $r_{i-1}$.
Otherwise, $r_i$.

The local transition function for $r_{n-1}$ is as follows.
If $h=0$ and the $n$ incoming edges from $Y$ encode the integer $2^{n-1}+n-1$, 
then $r_{n-2}$.
If $h =0$ and  the $n$ incoming edges from $Y$ encode the integer $n-1$, 
then operation $o_0$ applied to $r_{n-1}$ and $r_{n-2}$.
Otherwise, $r_{n-1}$.

Note that SyDS \cals{} is 6-symmetric; 
the local transition functions for nodes $r_i$ , $1 \leq i \leq n-1$, contain 6 symmetry classes.

For $0 \leq i < n$, we let $F_i$ be a Boolean function of $n-i-1$ variables, 
as follows.
$$F_i(x_{n-1}, x_{n-2}, \ldots , x_{i+1}) = (Q_ i x_i) 
\cdots (Q_1 x_1) (Q_0 x_0) f(X)$$
Table~\ref{tab:dag_syds_trans}
illustrates the dataflow as the constructed SyDS \cals{}  for $n =4$ goes through
its initial sequence of transitions.

%\bigskip

%\bigskip

\begin{table*}
\begin{center}
\begin{tabular}{|l | c c c c c c c c c c c |} \hline
\multicolumn{12}{|c|}{{\bf Initial Configuration Sequence of 
Constructed SyDS for Four Variables}}\\ \hline

$C_i$ &$r_3$ & $r_2$ & $r_1$ & $r_0$ & $W$ & $h$ & $y_4$ & $y_3$ & $y_2$ & $y_1$ & $y_0$ \\ \hline
$\calc_0$ & 0 & 0 & 0 & 0 & $--0--$ & 0 & 0 & 0 & 0 & 0 & 0 \\ 
$\calc_1$ & 0 & 0 & 0 & 0 & $W(0000)$ & 0 & 0 & 0 & 0 & 0 & 1 \\ 
$\calc_2$ & 0 & 0 & 0 & $f(0000)$ & $W(0001)$ & 0 & 0 & 0 & 0 & 1 & 0 \\ 
$\calc_3$ & 0 & 0 & 0 & $F_0(000)$ & $W(0010)$ & 0 & 0 & 0 & 0 & 1 & 1 \\ 
$\calc_4$ & 0 & 0 & $F_0(000)$ & $f(0010)$ & $W(0011)$ & 0 & 0 & 0 & 1 & 0 & 0 \\ 
$\calc_5$ & 0 & 0 & $F_0(000)$ & $F_0(001)$ & $W(0100)$ & 0 & 0 & 0 & 1 & 0 & 1 \\ 
$\calc_6$ & 0 & 0 & $F_1(00)$ & $f(0100)$ & $W(0101)$ & 0 & 0 & 0 & 1 & 1 & 0 \\ 
$\calc_7$ & 0 & $F_1(00)$ &$F_1(00)$ & $F_0(010)$ & $W(0110)$ & 0 & 0 & 0 & 1 & 1 & 1 \\ 
$\calc_8$ & 0 & $F_1(00)$ & $F_0(010)$ & $f(0110)$ & $W(0111)$ & 0 & 0 & 1 & 0 & 0 & 0 \\ 
$\calc_9$ & 0 & $F_1(00)$ & $F_0(010)$ & $F_0(011)$ & $W(1000)$ & 0 & 0 & 1 & 0 & 0 & 1 \\ 
$\calc_{10}$ & 0 & $F_1(00)$ & $F_1(01)$ & $f(1000)$ & $W(1001)$ & 0 & 0 & 1 & 0 & 1 & 0 \\ 
$\calc_{11}$ & 0 & $F_2(0)$ & $F_1(01)$ & $F_0(100)$ & $W(1010)$ & 0 & 0 & 1 & 0 & 1 & 1 \\ 
$\calc_{12}$ & $F_2(0)$& $F_2(0)$ & $F_0(100)$ & $f(1010)$ & $W(1011)$ & 0 & 0 & 1 & 1 & 0 & 0 \\ 
$\calc_{13}$ & $F_2(0)$& $F_2(0)$ & $F_0(100)$ & $F_0(101)$ & $W(1100)$ & 0 & 0 & 1 & 1 & 0 & 1 \\ 
$\calc_{14}$ & $F_2(0)$ & $F_2(0)$ & $F_1(10)$ & $f(1100)$ & $W(1101)$ & 0 & 0 & 1 & 1 & 1 & 0 \\ 
$\calc_{15}$ & $F_2(0)$ & $F_1(10)$ & $F_1(10)$ & $F_0(110)$ & $W(1110)$ & 0 & 0 & 1 & 1 & 1 & 1 \\ 
$\calc_{16}$ & $F_2(0)$ & $F_1(10)$ & $F_0(110)$ & $f(1110)$ & $W(1111)$ & 0 & 1 & 0 & 0 & 0 & 0 \\ 
$\calc_{17}$ & $F_2(0)$ & $F_1(10)$ & $F_0(110)$ & $F_0(111)$ & $W(0000)$ & 0 & 1 & 0 & 0 & 0 & 1 \\ 
$\calc_{18}$ & $F_2(0)$ & $F_1(10)$ & $F_1(11)$ & $f(0000)$ & $W(0001)$ & 0 & 1 & 0 & 0 & 1 & 0 \\ 
$\calc_{19}$ & $F_2(0)$ & $F_2(1)$ & $F_1(11)$ & $F_0(000)$ & $W(0010)$ & 0 & 1 & 0 & 0 & 1 & 1 \\ 
$\calc_{20}$ & $F_3$ & $F_2(1)$ & $F_1(11)$ & $f(0010)$ & $W(0011)$ & 1 & 1 & 0 & 1 & 0 & 0 \\ 
$\calc_{21}$ & $F_3$ & 0 & 0 & 0 & $W(0100)$ & 1 & 1 & 0 & 1 & 0 & 1 \\ \hline
\end{tabular}
\end{center}
\caption{Table illustrating the data flow for $n = 4$ 
as the DAG-SyDS \cals{} goes through the initial sequence of transitions}
\label{tab:dag_syds_trans}
\end{table*}

\bigskip
We now consider the correctness of the reduction.

Let $\beta$ be a tuple, possibly empty, of Boolean values. 
Let  $l(\beta)$ denote the degree of $\beta$,
i.e. the number of variables in $\beta$.
Let  $k(\beta)$ denote the integer encoded by $\beta$.
(If  $\beta$ is empty, then $k(\beta) = 0$.
For $\beta$ such that $l(\beta) < n$, 
let $j(\beta) = n - l(\beta) -1$,
and let $t(\beta) = (k(\beta)+1) 2^{n-l(\beta)} +n -l(\beta)$. 

For any $ t \geq 0$,
let  $X^t$ be the assignment to the variables $X$ of $f$ where 
$X^t(x_i) = \calc _t(y_i)$, $ 0 \leq i \leq n-1$.
From the proof in Theorem \ref{pro:long_phase_space_cycle}, 
the first $n$ nodes of $\cals$ undergo a phase space cycle whose length is $2^n$.
Thus, all $2^n$ possible assignments to $X$ occur during this cycle.
In particular, all $2^n$ assignments to $X$ are in the set $\{  X^t \, | \, 0 \leq t < 2^n \}$.
Specifically, for any assignment $\alpha$ to $X$, $X^{k(\alpha)} = \alpha$.


For any assignment $\alpha$ to $X$, 
let $W(\alpha)$ be the assignment of values to the nodes set $W$ 
where $w_i$ equals the value of clause $c_i$ for assignment $\alpha$, $ 0 \leq i \leq m-1$.
Then,  for all $t \geq 0$, $\calc_{t+1}[W] = W(X^t)$.

For any given tuple $\beta$ of  $n-i-1$ Boolean values 
corresponding to values of the variables $x_{n-1}, x_{n-2}, \ldots , x_{i+1}$,
note that $j(\beta) = i$.
Moreover, the value of $F_{j(\beta)}(\beta) = F_i(x_{n-1}, x_{n-2}, \ldots , x_{i+1})$
is determined by the $2^{i+1}$ assignments to $X$  occurring at times 
$k(\beta) 2^{n-l(\beta)}$ through $(k(\beta)+1) 2^{n-l(\beta)}-1$.

\smallskip
\noindent
{\bf Claim:} We claim that for all $j'$, $0 \leq j' < n$,
for all $\beta$ such that $j(\beta) = j'$,
$F_{j(\beta)}(\beta) =\calc_{t(\beta)}[r_{j(\beta)}]$.
We prove the claim by induction on $j'$.

\smallskip
\noindent
{\bf Basis Step:} Suppose that $j' = 0$.
Consider any $\beta$ such that $j(\beta) = 0$.
Then $l(\beta) = n-1$, $t(\beta) = 2k(\beta) + 3$,
and the variables in $\beta$ are $(x_{n-1}, x_{n-2}, \ldots , x_1)$.
By definition,
$$F_0(x_{n-1}, x_{n-2}, \ldots, x_1) 
= (Q_0 x_0) \, f(x_{n-1}, x_{n-2}, \ldots , x_1, x_0).$$
Thus,\\
$F_0(x_{n-1}, x_{n-2}, \ldots , x_1) ~=~$ \\
\hspace*{0.1in}
$f(x_{n-1}, x_{n-2}, \ldots , x_1, 0) \; o_0 \; 
f(x_{n-1}, x_{n-2}, \ldots , x_1, 1).$ \\
Let $\gamma^0 = (x_{n-1}, x_{n-2}, \ldots , x_1, 0)$ and\\
$\gamma^1 = (x_{n-1}, x_{n-2}, \ldots , x_1, 1)$.
Then, $F_0(\beta) =  
f(\gamma^0) \; o_0 \; f(\gamma^1)$.
Note that $\gamma^0 = X^{k(\gamma^0)}$ and $\gamma^1 = X^{k(\gamma^1)}$.
Also, $k(\gamma^0) = 2k(\beta)$ and $k(\gamma^1) = 2k(\beta)+1$.
Thus, at time $2k(\beta)+1$, $W = W[\gamma^0]$;
and at time $2k(\beta)+2$, $W = W[\gamma^1]$.
Since at time $2k(\beta)+1$, $h=0$, and $y_0 = 0$, 
the local transition function for $r_0$ sets $\calc_{2k(\beta)+2}[r_0]$ to be 
the {\em and} of the values of the incoming edges from $W$ at time $2k(\beta)+1$.
Thus, $\calc_{2k(\beta)+2}[r_0] = f(\gamma^0)$.
Since at time $2k(\beta)+2$, $h=0$, and $y_0 = 0$, 
the local transition function for $r_0$ sets $\calc_{2k(\beta)+3}[r_0]$ to be result of $0_i$
applied to $\calc_{2k(\beta)+2}[r_0]$ and
the {\em and} of the values of the incoming edges from $W$ at time $2k(\beta)+2$.
Thus, $\calc_{2k(\beta)+3}[r_0] = f(\gamma^0) \; o_i \; f(\gamma^1) = F_0(\beta)$.
This proves the claim for $j' = 0$.

\smallskip
\noindent
{\bf Inductive Step:} 
Now assume that the claim holds for a given value of $j'$, $0 \leq j' < n-1$.
We want to prove that the claim holds for $j'+1$.
Consider any $\beta$ such that $j(\beta) = j'+1$.
We need to show that $F_{j'+1}(\beta) = \calc_{t(\beta)}[r_{j'+1}]$.
We first note that $l(\beta) = n-j'-2$, 
$t(\beta) = (k(\beta) + 1)2^{j'+2}+j'+2$,
and the variables in $\beta$ are $(x_{n-1}, x_{n-2}, \ldots , x_{j'+2})$.
(If $j' = n-2$, then $\beta$ contains no variables.)
By definition, 
$F_{j'+1}(\beta) =  (Q_{j'+1} x_{j'+1}) \cdots (Q_1 x_1) (Q_0 x_0) f(X)$.
Thus,\\
$F_{j'+1}(x_{n-1}, x_{n-2}, \ldots , x_{j'+2}) = $\\
\hspace*{0.1in}
$F_{j'}(x_{n-1}, x_{n-2}, \ldots , x_{j'+2}, 0) \; o_{j'+1} \;$\\
\hspace*{0.1in} $F_{j'}(x_{n-1}, x_{n-2}, \ldots , x_{j'+2}, 1).$\\
Let $\gamma^0 = (x_{n-1}, x_{n-2}, \ldots , x_{j'+2}, 0)$
and $\gamma^1 = (x_{n-1}, x_{n-2}, \ldots , x_{j'+2}, 1)$.
Then, $F_{j'+1}(\beta) =$  
$F_{j'}(\gamma^0) \; o_{j'+1} \; F_{j'}(\gamma^1)$.
Note that
$t(\gamma^0) = (k(\gamma^0)+1)2^{j'+1}+j'+1$.
Since $k(\gamma^0)= 2k(\beta)$,
$t(\gamma^0) = (2k(\beta)+1)2^{j'+1}+j'+1 = k(\beta)2^{j'+2}+2^{j'+1}+j'+1$.
Also, $k(\gamma^1)= 2k(\beta)+1$, and
$t(\gamma^1) = (k(\gamma^1)+1)2^{j'+1}+j'+1= 
(2k(\beta)+2)2^{j'+1}+j'+1 =  (k(\beta)+1)2^{j'+2}+j'+1$.
By the inductive hypothesis,
$F_{j'}(\gamma^0) = \calc_{t(\gamma^0)}[r_{j(\gamma^0)}] = \calc_{t(\gamma^0)}[r_{j'}]$,
and $F_{j'}(\gamma^1) = \calc_{t(\gamma^1)}[r_{j(\gamma^1)}] = \calc_{t(\gamma^1)}[r_{j'}]$.

At time $t(\gamma^0) = k(\beta)2^{j'+2}+2^{j'+1}+j'+1$,
the $j'+2$ incoming edges to $r_{j'+1}$ from $Y$ encode the integer $2^{j'+1}+j'+1$,
so the local transition function for $r_{j'+1}$ evaluates to be the value of $r_{j'}$.
Thus, $\calc_{t(\gamma^0)+1}[r_{j'+1}] = \calc_{t(\gamma^0)}[r_{j'}] = F_{j'}(\gamma^0)$.
The local transition function for $r_{j'+1}$ has $r_{j'+1}$ retain its current value until 
$t(\gamma^1) = (k(\beta)+1)2^{j'+2}+j'+1$.
At time $t(\gamma^1) =  (k(\beta)+1)2^{j'+2}+j'+1$,
the $j'+2$ incoming edges to $r_{j'+1}$ from $Y$ encode the integer $j'+1$,
so the local transition function for $r_{j'+1}$ evaluates to be the result
of applying operation $o_{j'+1}$ to  $r_{j'+1}$ and $r_{j'}$.
Thus, 
$\calc_{t(\gamma^1)+1}[r_{j'+1}] = 
 \calc_{t(\gamma^1)}[r_{j'+1}]  \; o_{j'+1} \; \calc_{t(\gamma^1)}[r_{j'}] 
= F_{j'}(\gamma^0)  \; o_{j'+1} \; F_{j'}(\gamma^1) = F_{j'+1}(\beta)$.
However, note that $t(\beta) = (k(\beta) + 1)2^{j'+2}+j'+2 =  t(\gamma^1)+1$.
Thus, $F_{j'+1}(\beta) = \calc_{t(\beta)}[r_{j'+1}]$.
This completes the proof of the claim.

\smallskip

Note that each node in $Y$ and $W$ cycles with a period that divides $2^{n+1}$.
Since $\cald$ has all nodes in $Y$ equal to 0, 
and this only occurs at a time that is a multiple of $2^{n+1}$,
$\cald$ is reachable  iff it is reachable at a time that is a multiple of $2^{n+1}$.

Note that $\calc[h]=0$ and $\cald[h]=1$.  Furthermore, node $h$ is
stable with value 1 at time $2^n+n$, and $h$ is equal to 0 at all
prior times.  Thus $\cald$ is reachable iff it is reachable at a
time $t$ such that $t \geq 2^n+n$.  Considering node set $Y$, $\cald$
is reachable iff it is reachable  at a time that is a nonzero
multiple of $2^{n+1}$.

Note that every node of $\cals$ 
has the same value for every time that is a  nonzero multiple of $2^{n+1}$.
Thus, $\cald$ is reachable  iff it is reachable at time $2^{n+1}$,
i.e., iff $\cald = \calc_{2^{n+1}}$.

Recall that $\cald(h) = \calc_{2^{n+1}}(h)$ and $\cald[Y] = \calc_{2^{n+1}}[Y]$.
Since $\calc_{2^{n+1}-1}[Y]$ consists of all 1's,
each $\calc_{2^{n+1}}[w_j]$ equals 1 iff the corresponding clause $c_j$ contains a positive literal.
Thus, $\cald[W] = \calc_{2^{n+1}}[W]$.
The nodes in $R - \{r_{n-1}\}$ are stable with value 0 at time $2^n+n+1$,
and so have the same value in $\cald$ and $\calc_{2^{n+1}}$.
Thus, $\cald = \calc_{2^{n+1}}$ iff $\cald(r_{n-1}) = \calc_{2^{n+1}}(r_{n-1})$.
Since $\cald(r_{n-1}) = 1$, $\cald = \calc_{2^{n+1}}$ iff
$\calc_{2^{n+1}}(r_{n-1}) = 1$.

Note that node $r_{n-1}$ is stable at  time $2^n+n$,
so $ \calc_{2^{n+1}}(r_{n-1}) = \calc_{2^n+n}(r_{n-1})$.
Thus  $\cald$ is reachable iff $\calc_{2^n+n}(r_{n-1}) = 1$.
Let $\beta$ be the empty tuple.
As has been shown above,
$F_{j(\beta)}(\beta) =\calc_{t(\beta)}[r_{j(\beta)}]$.
Note that $l(\beta) = 0$, $k(\beta) = 0$, $j(\beta) = n-1$,
and $t(\beta) = (k(\beta)+1) 2^{n-l(\beta)} +n -l(\beta) = 2^n+n$. 
Thus, $\calc_{2^n+n}(r_{n-1}) = F_{n-1}$, which is the value of the given quantified Boolean formula $F$.
Thus $F$ is true iff $\cald$ is reachable from $\calc$.

\smallskip

The reduction from DAG $r$-symmetric to DAG symmetric SyDSs follows from Corollary \ref{cor:reachability_symmetric}.
\QED
