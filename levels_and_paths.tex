%\section{Number of Levels and Phase Space Paths}
\section{Some Properties of Phase Spaces of DAG-SyDSs}
\label{sec:bounded-levels}

In this section, we present some properties of the
phase spaces of DAG-SyDSs.
These properties are independent of the local functions
at the nodes of the DAG-SyDS.
%More refined versions of these properties when local
%functions are from special classes of Boolean functions
%are established in Section~\ref{sec:reachability}. 
For space reasons, proofs of several results 
in this section appear in the supplement.

Our first result points out that the phase spaces
of DAG-SyDSs may contain exponentially large cycles.

\smallskip
\begin{proposition}\label{pro:long_phase_space_cycle}
For any $n  > 1$, there is an $n$ node DAG-SyDS 
whose phase space graph is a cycle of length $2^n$.
\end{proposition}

\noindent
\textbf{Proof:}~ 
For a given $n > 1$ we construct the DAG-SyDS $S_n$ to be a counter, as follows.
The underlying graph contains $n$ levels, one node per level. 
For each node, there is an incoming edge from the nodes on each of the lower levels.
The transition function for each node is the function 
that retains the current value of the node if any of the lower order bits is 0,
and changes the value of the node if all of the lower order bits are 1.

Suppose a given configuration of $S_n$ is interpreted as encoding
an integer $k$, $0 \leq k < 2^n$.  Then the successor configuration
encodes the integer $k + 1 \mod 2^n$.  Thus, the phase space of
$S_n$ is a cycle of length $2^n$.  \QED

\smallskip

For any SyDS, any infinitely long phase space path consists of a
transient (possibly of length 0), followed by an infinite number
of repetitions of a basic cycle.  We now show that for any 
DAG-SyDS, the length of every phase space cycle is a power of 2.
Moreover, the lengths of the longest transient and the longest phase
space cycle are each bounded by an exponential function of the
number of levels in the underlying graph of the SyDS.
We begin with a lemma whose proof appears in the supplement.

\smallskip

\begin{lemma}\label{lem:all_inputs_stable}
For a given DAG-SyDS, and a given initial configuration, suppose
that the node values of all incoming edges to a given node $v$ are
stable at time $t$.  Then node $v$ is either alternating at time
$t$, or stable at time $t+1$.
\end{lemma}

\iffalse
%%%%%%%%%%%%%%%%%%%%%%%%%%%%%%%%%%%
\noindent
\textbf{Proof:}~ 
Given the stable value of each incoming variable, the local transition
function for node $v$ at any time $t'$, where $t' \geq t$ is a
function of only one variable: the self variable $v$.  The only
possible functions of a single variable are a constant function,
the identity function, or the complement function.  If the local
transition function for $v$ is a constant function, then the value
of $v$ can possibly change between $t$ and $t+1$, but for any $t'
\geq t+1$, remains unchanged.  If the local transition function for
$v$ is the identity function, then the value of $v$ never changes
from its value at time $t$.  If the local transition function for
$v$ is the complement function, then the value of $v$ alternates
between complementary values.  \QED
%%%%%%%%%%%%%%%%%%%%%%%%%%%%%%%%%%%
\fi

\begin{lemma}\label{lem:level_zero_nodes}
For a DAG-SyDS, every level 0 node is either 
alternating at time 0 or stable at time 1.
\end{lemma}
\noindent
\textbf{Proof:}~ 
 A level 0 node has no incoming edges,
so the result follows from Lemma \ref{lem:all_inputs_stable}.  \QED

We will also use the following result which is a slight restatement
of Proposition~1 in \cite{Chistikov-etal-2020}.


\begin{proposition}\label{pro:transient_fixed_point}
For a DAG-SyDS,
the length of a transient leading to a fixed point does not 
exceed the number of levels of the SyDS.
\end{proposition}

\iffalse
%%%%%%%%%%%%%%%%%%%%%%%%%%%%%%%%%%%%
\noindent
\textbf{Proof:}~ 
Consider a given SyDS $\cals$ and initial configuration $\calc$ that leads to a fixed point.
Let $L$ be the number of levels of $\cals$.
We argue that by induction on $i$, for each $i$, $0 \leq i < L$,
all the nodes in $\call_i'$ are stable at time $i+1$.

From Lemma \ref{lem:level_zero_nodes}, every level 0 node is either
alternating at time 0 or stable at time 1.  If any level 0 node is
alternating at time 0, then $\calc$ does not lead to a fixed point.
Thus, all level 0 nodes are stable at time 1.

Now assume that the induction hypotheses holds for $i$, $0 \leq i
< L-1$.  Let $v$ be a level $i+1$ node.  From the induction hypothesis,
all the incoming edges to $v$ are stable at time $i+1$.  From Lemma
\ref{lem:all_inputs_stable}, node $v$ is either alternating at time
$i+1$ or stable at time $i+2$.  If $v$ is alternating at time $i+1$,
then $\calc$ does not lead to a fixed point.  Thus, $v$ is stable
at time $i+2$.

Thus, all nodes of $\cals$ are stable at time $L$,
so  $\calc_L$ is the fixed point reached from $\calc$.
\QED
%%%%%%%%%%%%%%%%%%%%%%%%%%%%%%%%%%%%
\fi

\iffalse
%%%%%%%%%%%%%%%%%%%%%%%%%%%%%%%%%%%%%%

\begin{corollary}\label{cor:convergence}
For a DAG-SyDS, the convergence problem can be solved in polynomial time.
\end{corollary}
\noindent
\textbf{Proof:}~ Given a DAG-SyDS $\calc$ with $L$ and configuration $\calc$,
from Theorem \ref{thm:transient_fixed_point},
if $\calc$ converges, i.e., leads to a fixed point,
it does so after at most $L$ transitions.
In polynomial time, we can find $\calc_L$, and see if it is a fixed point.
\QED

\begin{theorem}\label{thm:convergence_guaranee}
The convergence guarantee problem for DAG-SyDSs is co-NP-complete,
even when restricted to dags with  three levels and 3-symmetric functions.
\end{theorem}
\noindent
\textbf{Proof:}~ 
We use a reduction from 3SAT, similar to that used in the proof of Theorem \ref{thm:fixed_point}.
The constructed SyDS $\cals$ has a level zero node for each variable, 
a level one node for each clause, and a single level two node.
We denote these three sets of nodes as $\{x_1, x_2, \ldots , x_n\}$,
$\{c_1,c_2, \ldots , c_m\}$, and $\{y\}$, respectively.

There is a directed edge from a  level zero node for a given variable 
to each of  the nodes for the clauses involving that variable.
There is a directed edge from each level one node to node $y$.
The local transition function for each level zero node is the identity function.
The local transition function for each level one node is the function that equals 1
 if the value of at least one of the incoming edges
makes the clause true. 
(This transition function is 3-symmetric because at least two of the literals are positive or negative.)
The local transition function for $y$ is the function $\bar{y} \, \wedge \, c_1 c_2 \cdots  c_{m}$.

Note that for every initial configuration, the level zero nodes are all stable at time 0,
and the level one nodes are all stable at time 1.

\smallskip
Suppose the given 3SAT problem instance is satisfiable.
Then any initial configuration of $\cals$ where the level zero nodes 
have values corresponding to a satisfying assignment
results in node $y$ being alternating at time 1.
Suppose the given 3SAT problem instance is unsatisfiable.
Then for every initial configuration, node $y$ is stable with value 0 at time 2.

\smallskip
Thus, $\cals$ reaches a fixed point from every initial configuration, 
i.e.,  convergence is guaranteed for $\cals$,
iff the given 3SAT problem instance is unsatisfiable.
\QED
%%%%%%%%%%%%%%%%%%%%%%%%%%%%%%%%%%%%%%
\fi

We can now state our result on the lengths of cycles and
transients in DAG-SyDSs.

\begin{theorem}\label{thm:levels_phase_space}
For a DAG-SyDS,
the length of every phase space cycle is a power of 2.
Moreover if the number of levels of a given acyclic SyDS is $L$,
then no phase space cycle is longer than $2^L$,
and no transient is longer than $2^L-1$.
\end{theorem}

\noindent
\textbf{Proof (idea):}~ We use induction on the number of levels.
The details appear in the supplement.

\iffalse
%%%%%%%%%%%%%%%%%%%%%%%%%%%%%%%%%%%%
\noindent
{\bf Basis}: Suppose that there is only one level.
Then the underlying  graph has no edges.
By Lemma \ref{lem:all_inputs_stable},
either all phase cycles are of length 1 or all are of length 2. 
Also, no transient is of length greater than 1.

{\bf Induction Step}: Suppose that the result holds for a given value of $L$,
and consider a DAG-SyDS \cals{} with $L+1$ levels (numbered as levels 0 through $L$).
Let $\cals_L$ be the DAG-SyDS with $L$ levels obtained 
by deleting $\call_L$ (the nodes in level $L$) from \cals{}.
Let \calc{} be a given configuration of \cals{}.

For non-negative integer $i$, 
recall that $C_i$ denotes the configuration of \cals{} at time $i$, 
when started from configuration \calc{}.
Since the underlying graph in $S$ has 
no edges from the level $L$ nodes to the nodes in $\cals_L$,
$C_i[\call_{L-1}']$ is the configuration at time $i$ 
when $S_L$ starts in configuration $C_0[\call_{L-1}']$.

From the induction hypothesis, 
when $\cals_L$  is started in configuration $C_0[\call_{L-1}']$,
the length of the phase space cycle that is reached is $2^k$, where $0 \leq k \leq L$,
and the length of the transient leading to this cycle is some value $j$ such that $j \leq 2^L-1$.

Let $v$ be a level $L$ node of \cals{}.
The sequence of values taken on by $v$ 
can be classified as belonging to one of the following three cases.

{\bf Case 1.} $C_{j+2^k}(v) = C_j(v)$:
Then, for every $i \geq j$, $C_{i+2^k}(v) = C_i(v)$.
Thus, the values of $C_i(v)$, where $i \geq j$, form a cycle, 
whose period divides (not necessarily properly) $2^k$.
So, after a transient of length at most $j$, where $j \leq 2^L-1$,
the value of node $v$ cycles with a cycle whose length is a power of 2 and is at most $2^L$.

{\bf Case 2}. $C_{j+2^k}(v) = \overline{C_j(v)}$ and $C_{j+2^{k+1}}(v) = C_j(v)$:
Then, for every $i \geq j$, $C_{i+2^{k+1}}(v) = C_i(v)$.
Thus, the values of $C_i(v)$, where $i \geq j$, form a cycle, 
whose period divides (not necessarily properly) $2^{k+1}$.
So, after a transient of length at most $2^L-1$,
the value of node $v$ cycles with a cycle whose length is a power of 2 and is at most $2^{L+1}$.

{\bf Case 3}. $C_{j+2^k}(v) = \overline{C_j(v)}$ and $C_{j+2^{k+1}}(v) = \overline{C_j(v)}$:
Then, for every $i \geq j + 2^k$, $C_{i+2^k}(v) = C_i(v)$.
Thus, the values of $C_i(v)$, where $i \geq j + 2^k$, form a cycle, 
whose period divides (not necessarily properly) $2^k$.
Note that  $\calc_{j+2^k}$ is part of this cycle.
So, after a transient of length at most $2^{L+1}-1$,
the value of node $v$ cycles with a cycle whose length is a power of 2 and is at most $2^L$.


Considering all the nodes in \cals{}, the length of the transient beginning at  \calc{} 
is at most $2^{k+1}-1$, 
and the length of the cycle reached from  \calc{} divides $2^{k+1}$.

Since this holds for all configurations \calc{}, 
the inductive hypothesis holds for \cals{}.
\QED
%%%%%%%%%%%%%%%%%%%%%%%%%%%%%%%%%%%%
\fi

We now present a result that presents matching 
lower bounds on cycle and transient lengths.  

\begin{theorem}\label{thm:path_length_lower_bounds}
For every $L  \geq 1$, there is a DAG-SyDS with $L$ levels
whose phase space contains a transient of length $2^L-1$,
leading to a cycle of length $2^L$.
\end{theorem}

\noindent
\textbf{Proof:}~ See supplement.

\iffalse
%%%%%%%%%%%%%%%%%%%%%%%%%%%%%%%%%%%%
We show this by constructing a DAG-SyDS that
incorporates the counter from Proposition \ref{pro:long_phase_space_cycle}
to produce a long cycle, and repeated applications of Case 3 from
the proof of Theorem \ref{thm:levels_phase_space} to produce a long
transient.

The constructed SyDS \cals{}  contains
$2L$ nodes, which we refer to as $X = \{x_0, x_1, \dots , x_{L-1}\}$ 
and $Y = \{y_0, y_1, \dots , y_{L-1}\}$.
For each $i$, nodes $x_i$ and $y_i$ will occur on level $i$.

The underlying graph of \cals{} has directed edges
$(x_i, x_j)$  and $(x_i, y_j)$ for each $i$ and $j$  such that $0 \leq i < j  < L$,
and also has directed edges $(y_i, y_{i+1})$ for each $i$ such that $0 \leq i < L-1$.

The local transition function for each node in $X$ is the function 
that retains the current value of the node if any of the incoming variables equals 0,
and changes the value of the node otherwise.
Thus, node set $X$ behaves as a counter,
similar to the nodes in the construction of Proposition \ref{pro:long_phase_space_cycle}.

The local transition function for node $y_0$ is the constant function 1.
The local transition function for each node $y_i$, $1 \leq i \leq L-1$,
is 1 iff $y_i = 1$ or $y_{i-1} = 1$ and the incoming variables $\{x_0, x_1, \dots , x_{i-1}\}$
encode the integer $2^i-2$.

Note that the $L$ nodes in $X$ form a phase space cycle of length $2^L$.

Consider the initial configuration \calc{}, where all nodes have the value 0.
It can be seen that by induction on $i$, for each node $y_i$, $0 \leq i \leq L-1$,
and each nonnegative $j$, 
$\calc_j(y_i)$ equals 0 if $j < 2^{i+1} -1$, and equals 1 if $j \geq 2^{i+1} -1$.
Since $y_{L-1}$ does not change from 0 to 1 until configuration $C_{2^L-1}$, 
the length of the transient from \calc{} is $2^L-1$.
\QED
%%%%%%%%%%%%%%%%%%%%%%%%%%%%%%%%%%%%
\fi
