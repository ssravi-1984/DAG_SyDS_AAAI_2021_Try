\section{Computationally Hard Problems for  Two-Level DAGs}
\label{sec:two-levels}


\begin{theorem}\label{thm:fixed_point}
The problem of determining whether a given DAG SyDS has a fixed point is NP-complete,
even when restricted to dags with two levels and maximum node degree 3.
\end{theorem}

\noindent
\textbf{Proof:}~ 
We use a reduction from 3SAT.
The constructed SyDS $S$ has a level zero node for each variable, 
and a level one node for each clause.
There is a directed edge from a  level zero node for a given variable 
to each of  the nodes for the clauses involving that variable.
The transition function for each level zero node is the identity function.
The transition function for each level one node is the function that equals 1
if the value of the node is 0, or if the value of at least one of the incoming edges
makes the clause true. 

\smallskip
Suppose the given 3SAT problem instance is satisfiable.
Then the configuration of $S$ where the level zero nodes 
have values corresponding to a satisfying assignment,
and the level one nodes all have value 1,
is a fixed point.

\smallskip

Suppose that SyDS $S$ has a fixed point.
In a fixed point, the level one nodes all have value 1,
and the values of the level zero nodes in each fixed point constitute a satisfying assignment.

\smallskip
Thus, $S$ has a fixed point iff the given 3SAT problem instance is satisfiable.

\smallskip
We observe that this reduction also establishes \#P-completeness.
\QED

\smallskip

\begin{theorem}\label{thm:predecessor_existence}
The predecessor existence problem for DAG SyDSs  is NP-complete,
even when restricted to dags with two levels and maximum node degree 3.
\end{theorem}

\noindent
\textbf{Proof:}~ 
We use a reduction from 3SAT.
The underlying graph of the constructed SyDS $S$ is the same as the underlying graph constructed 
in the proof of Theorem \ref{thm:fixed_point}.
The transition function for each level zero node is the constant function 1.
The transition function for each level one node is the function that equals 1
iff  the value of at least one of the incoming edges
makes the clause true. 
The configuration $C$ for the constructed predecessor existence problem instance 
is the configuration where all nodes have value 1.

\smallskip
Suppose the given 3SAT problem instance is satisfiable.
Then any configuration of $S$ where the level zero nodes 
have values corresponding to a satisfying assignment,
and the level one nodes have arbitrary values,
is a predecessor of configuration $C$.

\smallskip

Suppose that $C$ has a predecessor, say configuration $C_0$.
Since all the level one nodes in $C$ have value 1,
the values of the level zero nodes in $C_0$ constitute a satisfying assignment.

\smallskip
Thus, $C$ has a predecessor iff the given 3SAT problem instance is satisfiable.

\smallskip
Suppose the the given 3SAT problem instance contains $m$ clauses,
so that $S$ contains $q$ level one nodes.
Then, for any satisfiable assignment for the the given 3SAT problem instance,
$C$ contains $2^m$ predecessors.
Thus, the reduction also establishes \#P-completeness.
\QED
