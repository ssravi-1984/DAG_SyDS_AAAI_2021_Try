\subsection{DAG-SyDSs whose Local Functions are NCFs}
\label{ssec:ncfs}

%The main result of this section is that the reachability
%problem can be solved efficiently for DAG-SyDSs where each
%local function is a nested canalyzing function (NCF).
%This result is also obtained by establishing polynomial
%upper bounds on the lengths of transients and cycles in
%the phase space.
For space reasons, all the details regarding the efficient
solvability for this class of functions (including their definition) 
is given in the supplement (Section~\ref{asec:reach_ncf}).

\iffalse
%%%%%%%%%%%%%%%%%%%%%%%%%%%%%%
\begin{lemma}\label{lem:ncf_plus_two}
Let \cals{} be a DAG generalized-NCF SyDS.
For a given initial configuration,
let $v$ be a node such that the node values of all incoming edges to $v$
are  either stable or alternating at time $t$.
Then  $v$ is either stable or alternating at time $t+2$.
\end{lemma}

%\medskip
\noindent
\textbf{Proof:}~ Let $g_v$ be the generalized-NCF obtained from $f_v$  
by replacing those variables corresponding to incoming edges 
whose node values are stable at time $t$,
with these stable values,
shortening and deleting lines from the representation of the NCF as appropriate.. 
Note that $g_v$ determines the transitions of node $v$ for all times $t'$ such that $t' \geq t$.

Let $V'$ be the set of nodes with incoming edges to $v$, such that they are alternating at time $t$.
Let $\alpha$  be the assignment of Boolean values to $V'$ 
where each node  is assigned its value at time $t$.
Let $\overline{\alpha}$  be the assignment of Boolean values to $V'$ 
where each node  is assigned the complement of its value in $\alpha$.
Note that $\overline{\alpha}$ corresponds to the values of the nodes in $V'$ at time $t+1$.


The proof proceeds by a case analysis of $g_v$.

\noindent
\textbf{Case 1:}~ Suppose $g_v$ is a constant function.
Then node $v$ is stable at time $t+1$.

\noindent
\textbf{Case 2:}~ Suppose $g_v$ contains at least one canalyzing variable, 
but $v$ is not a canalyzing variable of $g_v$.
If $g_v(\overline{\alpha}) = g_v(\alpha)$, then node $v$ is stable at time $t+1$.
and if $g_v(\overline{\alpha}) \neq g_v(\alpha)$, 
then node $v$ is alternating at time $t+1$.

\noindent
\textbf{Case 3:}~ Suppose that $v$ is a canalyzing variable of $g_v$, 
but not is not the canalyzing variable in the first line of $g_v$.
We say that a given assignment to the nodes in $V'$ is {\bf dominating} 
if in the assignment, at least one of the variables occurring in a line prior to the line for $v$ in $g_v$
is equal to its canalyzing value.
Since assignments $\alpha$  and $\overline{\alpha}$ are complementary,
at least one of these assignments is dominating.
If assignment $\alpha$  is dominating, then $C_{t+1}(v)$ and $C_{t+3}(v)$ are equal,
so $v$ is either stable or alternating at time $t+1$.
If assignment $\overline{\alpha}$  is dominating, then $C_{t+2}(v)$ and $C_{t+4}(v)$ are equal,
so $v$ is either stable or alternating at time $t+2$.

\noindent
\textbf{Case 4:}~ Suppose that the first line of $g_v$ has $v$ as the canalyzing variable.

\begin{description}
\item{\textsf{Case 4A:}}~ The first line of $g_v$ is the following, for some canalyzing value $a$,
%\medskip

\noindent
\hspace*{1.1in} $v:~ a ~~\longrightarrow~~ a$


If $C_t(v)$ or $C_{t+2}(v)$ equals $a$,
then node $v$ is stable with value $a$ at time $t+2$.
If  $C_t(v)$  and $C_{t+2}(v)$ both equal $\overline{a}$,
then $C_{t+1}(v)$ also equals $\overline{a}$, and $v$ is stable with value $\overline{a}$ at time $t$

\item{\textsf{Case 4B:}}~ The first line of $g_v$ is the following, for some canalyzing value $a$,

\noindent
\hspace*{1.1in} $v:~ a ~~\longrightarrow~~ \overline{a}$

%\medskip

\begin{description}
\item{\textsf{Case 4B1:}}~ Suppose $C_t(v) = a$. 
Then $C_{t+1}(v) =  \overline{a}$. 
If $C_{t+2}(v) = a$, then node $v$ is alternating at time $t$.
So, suppose that $C_{t+2}(v) = \overline{a}$.
If $C_{t+3}(v) = \overline{a}$,
then node $v$ is stable with value $\overline{a}$ at time $t+1$.
If $C_{t+3}(v) = a$, then $C_{t+4}(v) = \overline{a}$,
so node $v$ is alternating at time $t+2$.

\item{\textsf{Case 4B2:}}~ Suppose $C_t(v) = \overline{a}$.
If $C_{t+1}(v) = a$, then $C_{t+2}(v) = \overline{a}$,
and $v$ is alternating at time $t$.
If $C_{t+1}(v) = \overline{a}$ and $C_{t+2}(v) = a$, then node $v$ is alternating at time $t+1$.
If $C_{t+1}(v) = \overline{a}$ and $C_{t+2}(v) =  \overline{a}$, then node $v$ is stable at time $t$.
\end{description} % of Case4B

\end{description}
\QED

\begin{theorem}\label{thm:NCF_phase_space}
For a DAG generalized-NCF \cals{},
the length of every phase space cycle is at most 2.
Moreover if the number of levels of  \cals{} is $L$,
no transient is longer than $2L-1$.
\end{theorem}
\noindent
\textbf{Proof:}~ 
From Lemma \ref{lem:level_zero_nodes},
each level 0 node is either stable or alternating at $t=1$.


From Lemma \ref{lem:ncf_plus_two},
for each subsequent level, each node at that level 
is either stable or alternating at most two steps after
all the nodes at lower levels have become stable or alternating.

Since each node of  \cals{} eventually becomes either stable or alternating,
 the length of every phase space cycle is at most 2.
Moreover, a node at a given level $j$ becomes either stable alternating
after at most $2j-1$ steps.
Since  \cals{} contains $L$ levels,
any directed path leading to a fixed point or a cycle of length 2
can contain at most $2L-1$ configurations.
\QED
%%%%%%%%%%%%%%%%%%%%%%%%%%%%%%
\fi
