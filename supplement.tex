%% File: supplement.tex

\appendix

\onecolumn

\begin{center}
\fbox{\textbf{Supplementary Material}}
\end{center}

\bigskip

\noindent
\underline{Paper title:}~
Synchronous Dynamical Systems on Directed Acyclic Graphs (DAGs): 
Complexity and Algorithms

\bigskip

\noindent
\section{Proofs of the Results in Section~\ref{sec:bounded-levels}}

\medskip

\iffalse
%%%%%%%%%%%%%%%%%%%%%%%%%%%
\noindent
\textbf{Statement and Proof of Proposition~\ref{pro:long_phase_space_cycle}}

\medskip

\underline{Statement of Proposition~\ref{pro:long_phase_space_cycle}:}~
For every $n  > 1$, there is an $n$ node DAG SyDS
whose phase space graph is a cycle of length $2^n$.

\medskip

\noindent
\textbf{Proof:}~
For a given $n > 1$ we construct the DAG SyDS $S_n$ to be a counter,
as follows.  The underlying graph contains $n$ levels, one node per
level.  For each node, there is an incoming edge from the nodes on
each of the lower levels.  The transition function for each node
is the function that retains the current value of the node if any
of the lower order bits is 0, and changes the value of the node if
all of the lower order bits are 1.

Suppose a given configuration of $S_n$ is interpreted as encoding
an integer $k$, $0 \leq k < 2^n$.  Then the successor configuration
encodes the integer $k + 1 \mod 2^n$.  Thus, the phase space of
$S_n$ is a cycle of length $2^n$.  \QED
%%%%%%%%%%%%%%%%%%%%%%%%%%%
\fi
