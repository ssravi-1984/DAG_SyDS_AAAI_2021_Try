\section{Linear Systems}
\label{sec:linear_functions}

We assume that a linear node transition function is represented by a linear formula,
consisting either of a constant, or a sum of distinct variables plus an optional constant.

We first consider the reachability problem.
For convenience, we assume that the question to be resolved
is whether the given initial configuration reaches the given final configuration in zero or more steps.
(If the problem formulation requires one or more steps,
the given problem instance can be modified by asking whether the configuration reached after one step
reaches the required final configuration after zero or more steps.)

\begin{theorem}\label{thm:linear_systems_reachability}
The reachability problem is solvable in polynomial time  for DAG linear SyDSs.
\end{theorem}
\noindent
\textbf{Proof:}~ 
We use an iterative algorithm, 
where the number of iterations does not exceed the number of levels of the given SyDS.

Let the given problem instance consist of SyDS \cals{} 
with underlying dag $G(V,E)$ and a linear local transition function $f_v$ for each node $v \in V$;
and two configurations \calc{} and \cald.
Let $L$ be the number of levels in dag $G$.
The algorithm consists of up to $L$ iterations, 
each of which we refer to as iteration $j$, $1 \leq j \leq L$.
Each iteration $j$, $1 \leq j \leq L$, processes an instance of the reachability problem, 
consisting of SyDS $\cals{}^j$
with underlying dag $G^j(V^j,E^j)$ 
and a linear local transition function $f^j_v$ for each node $v \in V^j$;
and two $\cals{}^j$ configurations $\calc^j$ and $\cald^j$.
Dag $G^j$ will contain at most  $L+1-j$ levels.
The reachability problem instance for each iteration will be constructed  
so that it has the same answer as the initial given problem instance.


For the first iteration, $\cals{}^1 = \cals{}$, $\calc^1 = \calc$ and $\cald^1 = \cald$.

Consider a given iteration $j$, $1 \leq j \leq L$.
If $\calc^j =\cald^j $, then the algorithm exits with answer  {\em YES}.
Otherwise, if $\cald^j $ is reachable from $\calc{}^j$,
it is reachable via a path requiring at least one transition.
The algorithm proceeds by constructing the configurations $\calc^j_1$ and $\calc^j_2$,
the one and two step successor configurations of $\calc^j$.
Consider $\call_0^j$, the set of level 0 nodes of $G^j$.
Note that from Lemma \ref{lem:level_zero_nodes},
at $t=1$ each node in $\call_0^j$ is either stable or alternating.
If $\calc^j_1[\call_0^j] \neq \cald^j[\call_0^j]$ and $\calc^j_2[\call_0^j] \neq \cald^j[\call_0^j]$,
then the algorithm exits with answer  {\em NO}.

So assume that $\calc^j_1[\call_0^j] = \cald^j[\call_0^j]$ or $\calc^j_2[\call_0^j] = \cald^j[\call_0^j]$,
If $G^j$ contains only one level, then $\cald^j = \cald^j[\call_0^j]$,
so the algorithm exits with answer  {\em YES}.

So assume that $G^j$ contains at least two levels.

We define configuration $\calc'^j$ as follows.
If $\calc^j_1[\call_0^j] = \cald^j[\call_0^j]$, 
then let $\calc'^j = \calc^j_1$, so that $\calc'^j_0 = \calc^j_1$;
otherwise let $\calc'^j = \calc^j_2$, so that $\calc'^j_0 = \calc^j_2$.
Note that $\cald^j$ is reachable from $\calc^j$ iff 
$\cald^j$ is reachable from $\calc'^j$.

The reachability problem instance for the next iteration $j+1$ is constructed with
$V^{j+1} = V^j -\call_0^j$,
$\cald^{j+1} =\cald^j[V^j -\call_0^j]$, and
$\calc^{j+1} =\calc'^j[V^j -\call_0^j]$.


For each node $v$ in $V^{j+1}$,
let $g_v$ denote the formula obtained from the formula for $f^j_v$ 
by replacing each occurrence in the formula of a node in $\call_0^j$
with the value of that node in $\calc'^j$;
and let $h_v$ denote the formula obtained from the formula for $f^j_v$ 
by replacing each occurrence in the formula of a node in $\call_0^j$
with the value of that node in $\calc'^j_1$.
%Let $\hat{g_v}$ denote the formula obtained from $g_v$
%by replacing each occurrence in the formula of a node $u$ in $V^{j+1}$
%with the formula $h_u$.
Let $\hat{h_v}$ denote the formula obtained from $h_v$
by replacing each occurrence in the formula of a node $u$ in $V^{j+1}$
with the formula $g_u$.
In the above formulas, 
we  simplify each formula by combining 
multiple occurrences of the same variable,
and combining multiple occurrences of additive constants.

For any $t \geq 0$, consider the configuration  $\calc'^j_t$ 
of $\cals{}^j$.
If $t$ is odd, then $\calc'^j_t[\call_0^j] = \calc'^j_1[\call_0^j]$;
so, for each $v \in V^{j+1} $,
formula $h_v$ gives the value of $v$ after one step from $\calc'^j_t$.
If $t$ is even, then $\calc'^j_t[\call_0^j] = \calc'^j[\call_0^j]$;
so, for each $v \in V^{j+1} $,
formula $g_v$ gives the value of $v$ after one step from $\calc'^j_t$,
and formula $\hat{h_v}$ gives the value of $v$ after two steps from $\calc'^j_t$.

For the remainder of the construction of $\cals^{j+1} $,
we consider the following two cases.

\noindent
\textbf{Case 1:}~
Every node in $\call_0^j$ is stable at $t =1$.
Thus, for all $t \geq 1$, $\calc'^j_t[\call_0^j] = \cald^j[\call_0^j]$.
In the reachability problem instance for the next iteration,
for each node $v \in V^{j+1}$,
we set the formula representing the node transition function $f^{j+1}_v$ 
to be the formula $g_v$.
(We observe that $g_v$ and $h_v$ are identical in this case.)
$E^{j+1}$ is the set of edges of $E^j$ having both endpoints in $V^{j+1}$.


\noindent
\textbf{Case 2:}~
$\call_0^j$ contains at least one node that is alternating at $t =1$.
Thus, for all $i \geq 0$,
$\calc'^j_{2i}[\call_0^j] = \cald^j[\call_0^j]$
and $\calc'^j_{2i+1}[\call_0^j] \neq \cald^j[\call_0^j]$.
Thus, $\cald^j$ is reachable from $\calc'^j$ iff
it is reachable via a path requiring an even number of transitions.
The node transition functions of $\cals{}^{j+1}$ will be constructed so that 
starting at initial configuration $\calc^{j+1}$ at time 0,
for all $ i \geq 0$,
$\calc^{j+1}_i = \calc'^j_{2i}[V^j -\call_0^j]$.
This property ensures that the reachability problem instance for iteration $j+1$
has the same answer as the reachability problem instance for iteration $j$.
For each node $v \in V^{j+1}$,
we let the formula representing the local transition function $f^{j+1}_v$ 
be the formula $\hat{h_v}$.
Edge set $E^{j+1}$ contains an incoming edge to $v$ from each node $u$
such that $u \neq v$ and $u$ occurs in formula $f^{j+1}_v$.

In constructing the problem instance for the next iteration,
each of the new node transition functions only contains variables
corresponding to nodes of the initial problem instance,
and so can be constructed and evaluated in time polynomial in
the size of the initial problem instance.
Thus, the algorithm runs in polynomial time.
\QED

We note that the algorithm in Theorem~\ref{thm:linear_systems_reachability} 
provides a possible template
for algorithms for other classes of Boolean functions.
For a given class of Boolean function,
if the two-step node functions required in Case 2 can be constructed and evaluated in time polynomial
in the original problem instance,
then the algorithm would run in polynomial time.
